\documentclass{../homework}

\usepackage{enumitem}

\newtheorem{problem}{Problem}
\newenvironment{solution}[1][\it{Solution}]{\textbf{#1. } }

\newcommand{\Prx}[1]{\Pr\left[#1\right]}
\newcommand{\ursample}{\overset{{\scriptscriptstyle\$}}{\leftarrow}}
\input{../name.tex}

\begin{document}

\homework{Homework \#2}
{Due: 18.03.23}{\profname}{\myname}

\begin{problem}
    Alice is encrypting with a one-time pad but unfortunately the message she
    wishes to send is double the length of the key. She solves this problem by
    concatenating the key  $k$ with itself and using this new key $k'=(k||k)$ to
    encrypt. 

    Write a program that breaks this scheme, that is, it achieves the following:
    It takes as input a ciphertext $c$ of length $2n$. You can expect it to be
    an encryption of concatenation of  two English-language words $(m_1||m_2)$
    using the one-time-pad. The words have equal length and there is no
    whitespace between them.  To produce the ciphertext, a key $k'=(k||k)$ has
    been used --- the key consists of concatenation of a $n$-bit bitstring $k$
    with itself. The program then finds $m_1$ and $m_2$.
    
    Consider the following ciphertext:
    $$c=\texttt{4A5C45492449552A5A47534D35525F20}$$
    (sixteen bytes in total, presented in hex).
    Figure out the plaintext using your program. The English-language wordlist
    is given in \texttt{wordlist.txt}.
\end{problem}
\begin{solution}
    The plaintext is either \texttt{answeredquestion} or
    \texttt{questionanswered}.
    
    It is possible to recover two keys by XORing either word with either half of
    the ciphertext. Because the key and both words have the same length, both
    keys will decrypt the other word in the other half of the ciphertext. As
    such, it is not possible to distinguish the used key from either option, and
    we cannot recover the original order of the messages without additional
    information.

    See \texttt{problem1.py} for the program's source code.
\end{solution}

\begin{problem}
    Consider the following encryption system. We fix a finite field
    $\mathbb{Z}_p$ where $p$ is a publicly known prime. From now on, we will
    work in this field. (Essentially this means that from now on, all addition,
    subtraction and multiplication and related operations will be considered
    modulo $p$. The fact that $p$ is a prime means that this structure is a
    field, that is, every nonzero element $a$ has one (and exactly one)
    multiplicative inverse element $a^{-1}$, that is, the element that satisfies
    $aa^{-1}=1$)
    
    Now, to encrypt an element $m\in \mathbb{Z}_p$ with the key $k\in
    \mathbb{Z}_p$, we compute $k+m$. That is, $Enc(k,m)=k+m$.
    
    Decryption is defined as $Dec(k,c)=c-k$.

\begin{itemize}
    \item Show that this encryption scheme has perfect secrecy.
    \item You are an adversary against this scheme in the following game. The
    challenger first samples a key at random from $\mathbb{Z}_p$. You submit a
    message $m_1\in \mathbb{Z}_p$ and get back its encryption $c_1=Enc(k,m_1)$.
    Then the challenger picks at random a message $m_2$  from $\mathbb{Z}_p$,
    encrypts it with the same key $k$ and gives you $c_2=Enc(k,m_2)$. Your task
    is to output $m_2$. Describe your strategy.
\end{itemize}
\end{problem}
\begin{solution}
\begin{itemize}
    \item We seek to prove perfect secrecy, i.e., that
    $$
        \Pr[c=c':k\ursample\Z_p, c'\leftarrow Enc(k, m_1)] =
        \Pr[c=c':k\ursample\Z_p, c'\leftarrow Enc(k, m_2)]
    $$
    for two messages $m_1, m_2\in\Z_p$.
    %Because $m, k\in\Z_p$ we have that $0\le m +k \le 2p - 2$.

    We have that $Enc(k, m) = c \Rightarrow k + m \equiv c \pmod{p} \Rightarrow
    c - m \equiv k \pmod{p}$ and so %there is only one $k = (c - m) \bmod{p}$
    %s.t. $Enc(k, m) = (k + m) \bmod{p} = c$.
    \begin{equation}\label{eq:uniqenq}
        \exists! k = (c - m) \bmod{p} : Enc(k, m) = (k + m) \bmod{p} = c.
    \end{equation}

    We can convince ourselves that this is true. Let $k, k'\in\Z_p$ s.t. $(k+m)
    \bmod{p} = c = (k'+ m) \bmod{p}$, with $k\neq k'$, and $x,y\in\Z$. Then
    \begin{alignat*}{2}
        &&\quad k + m + xp &= k' + m + yp\\
        \Leftrightarrow&& k &= k' + p(y - x)\\
        \Leftrightarrow&& k &\equiv k' \pmod{p}\\
        \Rightarrow&& k &= k'
    \end{alignat*}
    because $0\le k, k' < p$.

    From statement \ref{eq:uniqenq} we know that there exists a unique $k$
    satisfying $Enc(k, m)=c$. We also have that $\Pr[c=c':k\ursample\Z_p, c'
    \leftarrow Enc(k, m)]$ is the number of keys satisfying $Enc(k,m)=c$ divided
    by the total number of possible keys.

    We hence have
    $$
        \Pr[c=c':k\ursample\Z_p, c'\leftarrow Enc(k, m)]=
        \frac{|k\in\Z_p \text{ s.t. } Enc(k, m) = c|}{|\Z_p|}=\frac{1}{p}
    $$
    which holds for all $m,c'\in\Z_p$ and so
    $$
        \Pr[c=c':k\ursample\Z_p, c'\leftarrow Enc(k, m_1)] =
        \Pr[c=c':k\ursample\Z_p, c'\leftarrow Enc(k, m_2)]
    $$
    holds and the scheme has perfect secrecy by definition.

    \item For this game, we assume that we know $p$. We have
    $$
        m_1 + k \equiv c_1 \pmod{p}\quad\Leftrightarrow\quad
        c_1 - m_1 \equiv k \pmod{p}
    $$
    and because we know $m_1, c_1$, we can recover the adversary's key $k$ in
    the first part of the game. In the second part, we learn $c_2$ from the
    adversary, and we have
    $$
        m_2 + k \equiv c_2 \pmod{p}\quad\Leftrightarrow\quad
        c_2 - k \equiv m_2 \pmod{p}.
    $$

    Because the key $k$ is the same for both stages of the game, we can easily
    recover $m_2$ since
    $$
        m_2 = (c_2 - k) \bmod{p} = (c_2 - c_1 + m_1) \bmod{p}.
    $$

    As an example, suppose we have $p = 13, m_1 = 4, c_1 = 2, c_2 = 7$. Then
    we have $m_2 = (7 - 2 + 4) \bmod{13} = 9$.

    Let us verify this. We have $k = (2 - 4) \bmod{13} = 11$ and
    $c_2 = (9 + 11) \bmod{13} = 7$ holds.
\end{itemize}
\end{solution}

\end{document}