\documentclass{../homework}

\usepackage{etoolbox}
\usepackage{xcolor}
\hypersetup{
    colorlinks,
    linkcolor={red!50!black},
    citecolor={blue!50!black},
    urlcolor={blue!80!black}
}

%\usepackage{wrapfig,mathpartir,eurosym,algorithm,moreverb,tikz}
%\usetikzlibrary{graphs,graphdrawing,arrows.meta,calc}
%\usegdlibrary{trees}

\newtheorem{problem}{Problem}
\newenvironment{solution}[1][\it{Solution}]{\textbf{#1. } }

\input{../name.tex}

\newcommand{\rgets}{\stackrel{\mathdollar}\leftarrow}
\newcommand{\iseq}{\stackrel{?}=}

\newcommand{\Enc}{\text{Enc}}
\newcommand{\Dec}{\text{Dec}}
\newcommand{\tmac}{\text{TreeMAC}}

\begin{document}

\homework{Homework \#6}
{Due: 05.05.23}{\profname}{\myname}

\begin{problem}
    Alice sees that the problems of the hash-and-mac construction come from
    applying the hash function in a row. She knows about Merkle trees --- that
    it is possible to apply hash functions in a tree-like structure. Thus she
    proposes the following construction.
    
    Let $F:\{0,1\}^{2n} \rightarrow \{0,1\}^n$ be a compression function that
    is collision-resistant, first preimage resistant and second preimage
    resistant. (We omit the concrete $\tau$  and $\varepsilon$ because the
    specifics are not important, just imagine the $\tau$ to be very-very big
    and $\varepsilon$ to be very-very small.)
    
    Just constructing a binary tree would mean, however, that we can apply this
    constructions to messages whose length is $2^mn$ for some integer $m$.
    Thus, Alice proposes the following version of a tree construction.
    
    Let us have an input $x$, the length of which is a multiple on $n$. (This
    still sets some constraints on the length of the messages we can
    authenticate, but let us not worry about that.) Let this $x$ be chopped
    into $p$ blocks $(x_1,x_2,\dots,x_p)$ with $|x_i|=n$ --- that is, each
    block has $n$ bits. Given an input $x=(x_1,x_2,\dots,x_p)$ where
    $2^m<p\leq 2^{m+1}$ we define the function $V$ recursively the following
    way. First,
    $$
        V(x)=x,
    $$
    if $|x|=n$. That is, if $V$ is given a single block as an input, it will
    just output it without change. Secondly,
    $$
        V(x_1,x_2,\dots,x_p)=
        F(V(x_1,\dots,x_{2^{m}}),V(x_{2^{m}+1},\dots,x_p)),
    $$
    where $2^m<p\leq 2^{m+1}$. This can be thought of as a binary tree where
    the left half is a full binary tree and the right half is constructed
    recursively  using the same construction. For example, if $p=6$, then
    \begin{align*}
        V(x_1,x_2,x_3,x_4,x_5,x_6) &= F(V(x_1,x_2,x_3,x_4),V(x_5,x_6))\\
        &= F(F(V(x_1,x_2),V(x_3,x_4)), F(V(x_5),V(x_6)))\\
        &= F(F(F(V(x_1),V(x_2)), F(V(x_3),V(x_4))), F(x_5,x_6))\\
        &= F(F(F(x_1,x_2), F(x_3,x_4)), F(x_5,x_6)).
    \end{align*}
    
    Using this function $V$, Alice defines the TreeMAC for a $n$-bit key $k$ as
    follows.
    $$\text{TreeMAC}(k,x)=V(k,x_1,x_2,\dots,x_p),$$
    where $x=(x_1,x_2,\dots,x_p)$.
    
    Show that this scheme is not EF-CMA for messages whose length is a multiple
    of $n$. (that is, all the messages used in the game have the extra
    constraint that their length in bits must be a multiple of $n$)
\end{problem}
\begin{solution}
    Let $k$ be an $n$-bit key unknown to the adversary (us). We define messages
    $x_i := x_{i_1}||x_{i_2}||\dots||x_{i_p}$ with $|x_{i_j}| = n$, and
    $2^m < p \le 2^{m+1}$ for some non-negative integer $m$. Let $\mathcal{O}$
    be a $\tmac$-oracle with access to $k$. That is, $\mathcal{O}$ returns $t_i
    \leftarrow \tmac(k, x_i)$ for any received $x_i$.

    We will show that $\tmac$ does not have EF-CMA for messages whose length is
    a multiple of $n$.
    
    Let us fix $q = 2^{m+1} - 1$. We construct a message
    $x_1 = x_{1_1}||x_{1_2}||\dots||x_{1_q}$ of length $qn$, and send it to
    $\mathcal{O}$, who returns to us $t_1$. We then construct a message $x_2 =
    x_{2_1}||x_{2_2}$ of length $2n$, and a message 
    $x_3 = x_1||x_2 = x_{1_1}||x_{1_2}||\dots||x_{1_q}||x_{2_1}||x_{2_2}$ of
    length $(q+2)n$.

    We have by construction that
    \begin{align*}
        t_1 = \tmac(k, x_1) &= V(k, x_{1_1}, x_{1_2},\dots, x_{1_q})\\
        &=F(
            V(k, x_{1_1}, x_{1_2}, \dots, x_{1_{2^{m} - 1}}),
            V(x_{1_{2^{m}-1}}, \dots, x_{1_{2^{m+1}-1}})
        ),
    \end{align*}
    which is why we require that $q = 2^{m+1} - 1$.

    Because we have access to the compression function $F$, we can compute
    $h = F(x_{2_1}, x_{2_2})$. Let us also compute $t_3' = F(t_1, h)$.

    We then have
    \begin{align*}
        t_3' &= F(t_1, h)\\
        &= F(t_1, F(x_{2_1}, x_{2_2}))\\
        &= F(t_1, F(V(x_{2_1}), V(x_{2_2})))\\
        &= F(V(k, x_{1_1},\dots,x_{1_q}), V(x_{2_1}, x_{2_2}))\\
        &= V(k, x_{1_1},\dots,x_{1_q}, x_{2_1}, x_{2_2})\\
        &= \tmac(k, x_3) = t_3,
    \end{align*}
    and we have forged a valid $\tmac$ for $x_3$ without having access to $k$,
    or querying $\mathcal{O}$ about $x_3$. Hence, by definition, the scheme is
    not EF-CMA secure.\qed
\end{solution}

\begin{problem}
    Consider the following signature scheme. Let
    $F:\{0,1\}^n \rightarrow \{0,1\}^n$ be a $(\tau,\varepsilon)$-one-way
    function. That is, given $x$, $F(x)$ is easy to compute, but for all
    $\tau$-time adversaries $A$ we have that
    $$
    \Pr[F(x')=y|x\rgets \{0,1\}^n, y\gets F(x), x'\gets A(y)]\leq \varepsilon.
    $$
    
    The key generation algorithm is the following. First, one picks $2m$ random
    values from $\{0,1\}^n$, let those be
    $x_{1,0},x_{1,1},x_{2,0},x_{2,1},\dots,x_{m,0},x_{m,1}$. This is our secret
    key.
    
    The public key is computed as $y_{i,j}=F(x_{i,j})$ for all
    $i\in \{1,\dots,m\}$ and $j\in\{0,1\}$.
    
    To write it more visually, we can write it as 
    $$
    sk=
    \begin{pmatrix}
        x_{1,0} & x_{2,0} & \cdots & x_{m,0} \\
        x_{1,1} & x_{2,1} & \cdots & x_{m,1} 
    \end{pmatrix}
    $$
    and
    $$
    pk=
    \begin{pmatrix}
        y_{1,0} & y_{2,0} & \cdots & y_{m,0} \\
        y_{1,1} & y_{2,1} & \cdots & y_{m,1}
    \end{pmatrix}.
    $$
    
    This is a signature scheme that is designed to provide signatures for
    messages that have exactly $m$ bits and that is usable only once. The
    signature of a $m$-bit message $M$ with bits $M_1,M_2,\dots,M_m$, is the
    message $(x_{1,M_1},x_{2,M_2},\dots,x_{m,M_m})$. To verify the signature
    $(c_1,\dots,c_m)$ of a message $M$ with bits $M_1,M_2,\dots,M_m$, one
    computes $F(c_1),F(c_2),\dots,F(c_m)$ and checks whether
    $F(c_i) \iseq y_{i,M_i}$ for all $i\in \{1,\dots,m\}$. If for all
    $i\in \{1,\dots,m\}$ we have that $F(c_i) = y_{i,M_i}$, the verification
    algorithm outputs $1$, else it outputs $0$.
    
    Show that this scheme does not satisfy EF-CMA for signatures with the added
    constraint that all the messages in the EF-CMA game have to be $m$ bits
    long.
\end{problem}
\begin{solution}
    Intuitively, we can see that if a new key is generated for every signature,
    the scheme works analogously to a OTP and is practically unbreakable.

    Let us show that this scheme does not satisfy EF-CMA for signatures of
    length $m$-bits, assuming that the same key $sk$ is used for signing
    multiple messages.

    Let $sk$ be the secret key known to the signing oracle $\mathcal{O}$, but
    not known to the adversary (us). $\mathcal{O}$ uses the same secret key
    $sk$ for all queries within its lifetime, and returns
    $c_i = (c_{i_1}, \dots, c_{i_m})$ for any received message $M_i$ with bits
    $M_{i_1}, M_{i_2}, \dots, M_{i_m}$. 

    We construct two $m$-bit messages
    \begin{enumerate}
        \item the $0$-message $M_0$ with bits $M_{0_j}=0,j\in\{1,2,\dots, m\}$,
        \item the $1$-message $M_1$ with bits $M_{1_j}=1,j\in\{1,2,\dots, m\}$,
    \end{enumerate}
    and send them to $\mathcal{O}$. In return, we receive $c_0, c_1$, with
    \begin{enumerate}
        \item $c_0 = (c_{0_1}, \dots, c_{0_m}) =
            (x_{1, M_{0_1}}, x_{2, M_{0_2}}, \dots, x_{m, M_{0_m}}) =
            (x_{1, 0}, x_{2, 0}, \dots, x_{m,0})$,
        \item $c_1 = (c_{1_1}, \dots, c_{1_m}) =
            (x_{1, M_{1_1}}, x_{2, M_{1_2}}, \dots, x_{m, M_{1_m}}) =
            (x_{1, 1}, x_{2, 1}, \dots, x_{m,1})$,
    \end{enumerate}
    and we can reconstruct the secret key
    $$
    sk_{rec} =
    \begin{pmatrix}
        c_0 \\
        c_1
    \end{pmatrix}
    =
    \begin{pmatrix}
        x_{1,0} & x_{2,0} & \cdots & x_{m,0} \\
        x_{1,1} & x_{2,1} & \cdots & x_{m,1} 
    \end{pmatrix}.
    $$

    We now construct a new message $M_{2_j}\rgets\{0,1\}$ with
    $j\in\{1,\dots,m\}$, and, without querying $\mathcal{O}$, we produce the
    signature $c_2$ with
    $$
        c_2 = (x_{1,M_{2_1}}, x_{2,M_{2_2}},\dots,x_{m, M_{2_m}}).
    $$

    We can do this because $M_{2_j}\in\{0, 1\}$ and we now know $x_{j, k}$ for
    all $j\in\{1,\dots,m\}$ and $k\in\{0,1\}$. In the extremely unlikely event
    that $M_2 = M_0$ or $M_2 = M_1$, we flip the first bit of the message to
    produce a fresh\footnotemark{} message.
    \footnotetext{A message that has not been sent to $\mathcal{O}$.}

    For all $j\in\{1,\dots,m\}$, we now have
    $$
        F(c_{2_j}) = F(x_{j, M_{2_j}}) = y_{j, M_{2_j}}
    $$
    by construction of the public key and we have produced a message $M_2$ with
    a verifiable (valid) signature $c_2$ without having queried $\mathcal{O}$
    for $M_2$. Hence, by definition, the scheme is not EF-CMA secure.\qed
\end{solution}
\end{document}
