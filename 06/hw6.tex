\documentclass{../homework}

\usepackage{etoolbox}
\usepackage{xcolor}
\hypersetup{
    colorlinks,
    linkcolor={red!50!black},
    citecolor={blue!50!black},
    urlcolor={blue!80!black}
}

%\usepackage{wrapfig,mathpartir,eurosym,algorithm,moreverb,tikz}
%\usetikzlibrary{graphs,graphdrawing,arrows.meta,calc}
%\usegdlibrary{trees}

\newtheorem{problem}{Problem}
\newenvironment{solution}[1][\it{Solution}]{\textbf{#1. } }

\input{../name.tex}

\newcommand{\rgets}{\stackrel{\mathdollar}\leftarrow}
\newcommand{\iseq}{\stackrel{?}=}

\newcommand{\Enc}{\text{Enc}}
\newcommand{\Dec}{\text{Dec}}
\newcommand{\tmac}{\text{TreeMAC}}

\begin{document}

\homework{Homework \#6}
{Due: 05.05.23}{\profname}{\myname}

\begin{problem}
    Alice sees that the problems of the hash-and-mac construction come from
    applying the hash function in a row. She knows about Merkle trees --- that
    it is possible to apply hash functions in a tree-like structure. Thus she
    proposes the following construction.
    
    Let $F:\{0,1\}^{2n} \rightarrow \{0,1\}^n$ be a compression function that
    is collision-resistant, first preimage resistant and second preimage
    resistant. (We omit the concrete $\tau$  and $\varepsilon$ because the
    specifics are not important, just imagine the $\tau$ to be very-very big
    and $\varepsilon$ to be very-very small.)
    
    Just constructing a binary tree would mean, however, that we can apply this
    constructions to messages whose length is $2^mn$ for some integer $m$.
    Thus, Alice proposes the following version of a tree construction.
    
    Let us have an input $x$, the length of which is a multiple on $n$. (This
    still sets some constraints on the length of the messages we can
    authenticate, but let us not worry about that.) Let this $x$ be chopped
    into $p$ blocks $(x_1,x_2,\dots,x_p)$ with $|x_i|=n$ --- that is, each
    block has $n$ bits. Given an input $x=(x_1,x_2,\dots,x_p)$ where
    $2^m<p\leq 2^{m+1}$ we define the function $V$ recursively the following
    way. First,
    $$
        V(x)=x,
    $$
    if $|x|=n$. That is, if $V$ is given a single block as an input, it will
    just output it without change. Secondly,
    $$
        V(x_1,x_2,\dots,x_p)=
        F(V(x_1,\dots,x_{2^{m}}),V(x_{2^{m}+1},\dots,x_p)),
    $$
    where $2^m<p\leq 2^{m+1}$. This can be thought of as a binary tree where
    the left half is a full binary tree and the right half is constructed
    recursively  using the same construction. For example, if $p=6$, then
    \begin{align*}
        V(x_1,x_2,x_3,x_4,x_5,x_6) &= F(V(x_1,x_2,x_3,x_4),V(x_5,x_6))\\
        &= F(F(V(x_1,x_2),V(x_3,x_4)), F(V(x_5),V(x_6)))\\
        &= F(F(F(V(x_1),V(x_2)), F(V(x_3),V(x_4))), F(x_5,x_6))\\
        &= F(F(F(x_1,x_2), F(x_3,x_4)), F(x_5,x_6)).
    \end{align*}
    
    Using this function $V$, Alice defines the TreeMAC for a $n$-bit key $k$ as
    follows.
    $$\text{TreeMAC}(k,x)=V(k,x_1,x_2,\dots,x_p),$$
    where $x=(x_1,x_2,\dots,x_p)$.
    
    Show that this scheme is not EF-CMA for messages whose length is a multiple
    of $n$. (that is, all the messages used in the game have the extra
    constraint that their length in bits must be a multiple of $n$)
\end{problem}
\begin{solution}
    Let $k$ be an $n$-bit key unknown to the adversary (us). We define messages
    $x_i := x_{i_1}||x_{i_2}||\dots||x_{i_p}$ with $|x_{i_j}| = n$, and
    $2^m < p \le 2^{m+1}$ for some non-negative integer $m$. Let $\mathcal{O}$
    be a $\tmac$-oracle with access to $k$. That is, $\mathcal{O}$ returns $t_i
    \leftarrow \tmac(k, x_i)$ for any received $x_i$.

    We will show that $\tmac$ does not have EF-CMA for messages whose length is
    a multiple of $n$.
    
    Let us fix $q = 2^{m+1} - 1$. We construct a message
    $x_1 = x_{1_1}||x_{1_2}||\dots||x_{1_q}$ of length $qn$, and send it to
    $\mathcal{O}$, who returns to us $t_1$. We then construct a message $x_2 =
    x_{2_1}||x_{2_2}$ of length $2n$, and a message 
    $x_3 = x_1||x_2 = x_{1_1}||x_{1_2}||\dots||x_{1_q}||x_{2_1}||x_{2_2}$ of
    length $(q+2)n$.

    We have by construction that
    \begin{align*}
        t_1 = \tmac(k, x_1) &= V(k, x_{1_1}, x_{1_2},\dots, x_{1_q})\\
        &=F(
            V(k, x_{1_1}, x_{1_2}, \dots, x_{1_{2^{m} - 1}}),
            V(x_{1_{2^{m}-1}}, \dots, x_{1_{2^{m+1}-1}})
        ),
    \end{align*}
    which is why we require that $q = 2^{m+1} - 1$.

    Because we have access to the compression function $F$, we can compute
    $h = F(x_{2_1}, x_{2_2})$. Let us also compute $t_3' = F(t_1, h)$.

    We then have
    \begin{align*}
        t_3' &= F(t_1, h)\\
        &= F(t_1, F(x_{2_1}, x_{2_2}))\\
        &= F(t_1, F(V(x_{2_1}), V(x_{2_2})))\\
        &= F(V(k, x_{1_1},\dots,x_{1_q}), V(x_{2_1}, x_{2_2}))\\
        &= V(k, x_{1_1},\dots,x_{1_q}, x_{2_1}, x_{2_2})\\
        &= \tmac(k, x_3) = t_3,
    \end{align*}
    and we have forged a valid $\tmac$ for $x_3$ without having access to $k$,
    or querying $\mathcal{O}$ about $x_3$. Hence, by definition, the scheme is
    not EF-CMA secure.\qed
\end{solution}
\end{document}
