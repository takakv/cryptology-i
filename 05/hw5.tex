\documentclass{../homework}

\usepackage{etoolbox}
\usepackage{xcolor}
\hypersetup{
    colorlinks,
    linkcolor={red!50!black},
    citecolor={blue!50!black},
    urlcolor={blue!80!black}
}

\newtheorem{problem}{Problem}
\newenvironment{solution}[1][\it{Solution}]{\textbf{#1. } }

\newcommand{\Prx}[1]{\Pr\left[#1\right]}
\newcommand{\ursample}{\overset{{\scriptscriptstyle\$}}{\leftarrow}}
\newcommand{\st}{\text{ such that }}
\input{../name.tex}

\newcommand{\rgets}{\stackrel{\mathdollar}\leftarrow}
\newcommand{\Enc}{\text{Enc}}
\newcommand{\Dec}{\text{Dec}}

\begin{document}

\homework{Homework \#5}
{Due: 25.04.23}{\profname}{\myname}

\begin{problem}
    Consider the following hash function $H_D$. Let $G$ be a group where the
    discrete logarithm is hard and let $g$ be a fixed generator in that group.
    Let the group size of $G$ be $k$. Let $k$ have $n$ bits, with
    $2^{n-1}<k<2^n$. $k$ is publicly known.

    More specifically, the hardness of the discrete logarithm means that we have
    for any $\tau$-time adversary $A$ that
    $$\Pr[y=g^{a'} :a \rgets\{1,\dots,|G|-1\}, y\gets g^a,
    a'\gets A(G,k,g,y)]\leq \varepsilon.$$
    
    Here $\tau$ is not very big (if you prefer more formal words, $\tau$ is
    bounded by some polynomial function in $n$) and $\varepsilon$ is
    very-very-very small (if you prefer more formal wording, $\varepsilon$ is
    exponentially small in $n$).
    
    Let $iv$ be a  fixed $n$-bit integer. %Let $g$ a fixed generator for $G$.
    
    This is a hash function built analogously to the iterated hash construction.
    
    Now, to compute the $H_{D,iv}$ for a fixed $iv$ we do the following.
    Given an input $m$ with $\ell n$ bits, we chop it into $n$-bit blocks
    $m_1,m_2,\dots,m_{\ell}$. We can consider all  $m_i$ as  $n$-bit integers. 
    
    Now, let $h_0:=g^{iv}$. Let
    $h_i:=h_{i-1}^{m_i}$ for all $i\in \{1,\dots,\ell\}$. We define
    $H_{D,iv}(m):=h_{\ell}$. Essentially this is very similar to the $H_{IH}$
    construction, but there we used  $h_i:=F(h_{i-1},m_i)$, but here we use
    $h_i:=h_{i-1}^{m_i}$.

    \begin{enumerate}
        \item  Is this scheme collision-resistant for messages with the same
        length? That is, does it hold that for any $\tau'$-time adversary $A$,
        you have that $$\Pr[m\neq m'\& H_{D,iv}(m)=H_{D,iv}(m')\&  |m|=|m'| :$$
        $$
        iv \rgets\{1,\dots,|G|-1\},  (m,m')\gets A(G,g,k,iv) ]\leq\varepsilon'
        $$
        where $\varepsilon'$ is very small  and $\tau'$ is not too big? (roughly
        speaking, $\varepsilon'$ should be not too different from $\varepsilon$
        and $\tau'$ should not be too different from $\tau$)
        
        If yes, explain why, preferably reducing the security to hardness of the
        DLOG. If no, provide an attack.

        \item Is this scheme preimage-resistant? More specifically, does it hold
        for any $\tau'$-time adversary $A$, that
        $$
            \Pr[H_{D,iv}(x)=y :iv \rgets\{1,\dots,|G|-1\},y \rgets G, x\gets
            A(G,g,k,iv,y) ]\leq \varepsilon'
        $$
        where $\varepsilon'$ is very small  and $\tau'$ is not too big? (roughly
        speaking, $\varepsilon'$ should be not too different from $\varepsilon$
        and $\tau'$ should not be too different from $\tau$)
        If yes, explain why, preferably reducing the security to hardness of the
        DLOG. If no, provide an attack.
        
        \item Is this scheme second preimage-resistant? More specifically, fix
        $\ell$ to be some integer larger than or equal to $2$. Does it hold that
        $$\Pr[m\neq m'\& H_{D,iv}(m)=H_{D,iv}(m')\&  |m|=|m'| :$$
        $$
            iv \rgets\{1,\dots,|G|-1\},m\gets \{0,1\}^{\ell n}, m'\gets
            A(G,g,k,iv,m) ]\leq \varepsilon'
        $$
        where $\varepsilon'$ is very small  and $\tau'$ is not too big? (roughly
        speaking, $\varepsilon'$ should be not too different from $\varepsilon$
        and $\tau'$ should not be too different from $\tau$)
        If yes, explain why, preferably reducing the security to hardness of the
        DLOG. If no, provide an attack.
        \end{enumerate}

        Hint: Generally, if you want to reduce the security of a scheme $A$ to
        the hardness of some other scheme/game $B$, you often need to play a
        middleman between the challenger of $B$ and the adversary of $A$.
        The argument goes approximately like this. Assume by contradiction that
        there exists an efficient adversary against $A$, let's call it $Adv_A$.
        Now we want to use $Adv_A$ as a tool or subroutine to break the game of
        $B$ when talking to the challenger in game $B$. We might have to process
        the messages that we give to $Adv_A$ so they would have the right format
        and  may also have to do things with the messages we get back from
        $Adv_A$ to compute what we need to compute for the game $B$.
\end{problem}
\begin{solution}
    The hashing scheme is built as follows:
    $$
        h_0 = g^{iv},\quad
        h_1 = h_0^{m_1} = \left(g^{iv}\right)^{m_1} = g^{iv \cdot m_1},\quad
        h_2 = h_1^{m_2} = \left(g^{iv\cdot m_1}\right)^{m_2} =
        g^{iv\cdot m_1 \cdot m_2},\quad \dots,
    $$
    and more generally,
    \begin{equation}\label{eq:1}
        H_{D, iv}(m) = h_l = g^{iv\cdot \Pi_{i = 1}^l m_i}.
    \end{equation}

    \begin{enumerate}
        \item Let us show that the scheme is not collision-resistant in $G$ for
        messages of the same length. That is, the adversary can craft $m'\neq m$
        such that $|m'| = |m|$ and with $H_{D, iv}(m') \neq H_{D, iv}(m)$.
        \begin{enumerate}
            \item From \eqref{eq:1} we see that breaking collision resistance is
            equivalent to the adversary finding $m' \neq m$ such that
            \begin{equation*}\label{eq:2}
                \Pi_{i=1}^l m_i' = \Pi_{i=1}^l m_i,
            \end{equation*}
            because
            $$
                \Pi_{i=1}^l m_i' = \Pi_{i=1}^l m_i \Rightarrow
                g^{iv\cdot \Pi_{i = 1}^l m'_i} = g^{iv\cdot \Pi_{i = 1}^l m_i}
                \Rightarrow h'_l = h_l \Rightarrow H_{D, iv}(m') = H_{D, iv}(m).
            $$
            \item Because multiplication is commutative, we have
            $$
                \Pi_{i=1}^l x_i = \Pi_{i=1}^{l} x_{l - i + 1}.
            $$
            and so, if we pick $m'$ such that $m'_i = m_{l - i + 1}$, clearly
            $m'\neq m$ and $\Pi_{i=1}^l m_i' = \Pi_{i=1}^l m_i$, with
            $|m'|=|m|$.
        \end{enumerate}

        As such, from (b) we can construct a message that satisfies (a), and so
        the scheme is not collision resistant for messages of the same length.
        \qed

        \item Let us show that the scheme is first preimage-resistant in $G$.

        Assume that there exists and adversary who, given $y\rgets G$, can
        output $x$ s.t. $H_{D, iv}(x) = y$.

        From \eqref{eq:1}, this means that the adversary can find $x$ s.t.
        $$
            H_{D,iv}(x)=h_l=g^{iv\cdot\Pi_{i = 1}^lx_i} = y.
        $$
        
        If we fix $e_x = iv\cdot\Pi_{i = 1}^lx_i$, we have that the adversary
        can produce $x$ s.t. $g^{e_x}=y$. This means that given $g, y$, the
        adversary can find $e_x$, and construct $x$ to match. Finding $e_x$ is
        equivalent to the discrete logarithm problem which is defined to be hard
        in $G$. As such, because the adversary is bound by some polynomial
        function, we know that the adversary cannot efficiently solve the DLOG,
        which contradicts our initial assumption.

        Therefore, by contradiction, we showed that there can be no polynomially
        bound adversary to break the first-preimage resistance of $H_{D, iv}$ in
        $G$.\qed

        \item Let us show that the scheme is not second preimage-resistant in
        $G$.
    \end{enumerate}
\end{solution}
\end{document}
