\documentclass{../homework}

\usepackage{etoolbox}
\usepackage{xcolor}
\hypersetup{
    colorlinks,
    linkcolor={red!50!black},
    citecolor={blue!50!black},
    urlcolor={blue!80!black}
}

\newtheorem{problem}{Problem}
\newenvironment{solution}[1][\it{Solution}]{\textbf{#1. } }

\input{../name.tex}

\newcommand{\rgets}{\stackrel{\mathdollar}\leftarrow}
\newcommand{\iseq}{\stackrel{?}=}

\newcommand{\Enc}{\text{Enc}}
\newcommand{\Dec}{\text{Dec}}
\newcommand{\tmac}{\text{TreeMAC}}

\begin{document}

\homework{Homework \#7}
{Due: 16.05.23}{\profname}{\myname}

\begin{problem}
    It is possible to consider encryption/decryption algorithms that use hash
    functions as part of their design. Thus it makes sense to consider IND-CPA
    security in the random oracle model. Write down the definition of IND-CPA
    security in the random oracle model (for symmetric encryption schemes).
\end{problem}
\begin{solution}
    \iffalse
    Let us remind ourselves of the definition of IND-CPA for symmetric schemes
    in the standard model.

    An encryption scheme $(KG, E, D)$ consisting of a key-generation algorithm
    $KG$, an encryption algorithm $E$, and a decryption algorithm $D$ is
    $(\tau, \epsilon)$-IND-CPA (indistinguishable under chosen plaintext
    attacks) if for any $\tau$-time algorithm $A$ we have that
    \begin{multline*}
        |\Pr[b = 1 : k \leftarrow KG(), (m_0, m_1) \leftarrow A^{E(k,\cdot)}(),
        c \leftarrow E(k, m_0), b \leftarrow A^{E(k,\cdot)}(c)]\\ -
        \Pr[b = 1 : k \leftarrow KG(), (m_0, m_1) \leftarrow A^{E(k,\cdot)}(),
        c \leftarrow E(k, m_1), b \leftarrow A^{E(k,\cdot)}(c)]| \le \epsilon.
    \end{multline*}
    \begin{center}
        (Here we quantify only over algorithms A that output $(m_0, m_1)$ with
        $|m_0| = |m_1|$.)
    \end{center}
    \fi
    Let there be a random oracle $H: M \rightarrow N$, and a one way function
    $f^H: M'\rightarrow N'$ using this oracle. Let $f^H$ be such that for any
    $\tau$-time oracle algorithm $A^H$ and any security parameter $\eta$, there
    exists a negligible function $\mu$ with
    $$
        \Pr[f^H(x) = f^H(x') : H \rgets \mathrm{Fun}_{M\rightarrow N},
        x \rgets M', y \leftarrow f(x), x' \leftarrow A^H (1^\eta, y)]
        \le \mu(\eta).
    $$

    Let ($KG, E, D)$ be an encryption scheme consisting of a key-generation
    algorithm $KG$, an encryption algorithm $E$ and a decryption algorithm $D$.
    To consider this scheme in the random oracle model (ROM), we must model $E$
    with access to the random oracle $H$. We hence have $c\leftarrow E^H(k, m)$
    for any key $k\leftarrow KG()$ and some message $m$.

    For IND-CPA in the standard model, there is a $\tau$-time adversary
    $A^{E(k, \cdot)}$ with access to some encryption oracle $E(k, \cdot)$. In
    the ROM however, the adversary must also have access to the random oracle
    $H$. Therefore, in the ROM, we consider some adversary
    $A^{H, E^H(k, \cdot)}$.

    We can write the adversary submitting a message to the encryption oracle as
    $$
        k \leftarrow KG(), H \rgets \mathrm{Fun}_{M\rightarrow N},
        m \leftarrow A^{H, E^H(k, \cdot)}(), c\leftarrow E^H(k, m).
    $$

    The definition of IND-CPA for symmetric schemes in the ROM then becomes:

    \textit{Given a random oracle $H \rgets \mathrm{Fun}_{M\rightarrow N}$
    (where $M$ and $N$ may depend on some security parameter $\eta$), we call a
    symmetric encryption scheme $(KG, E, D)$ $(\epsilon, \tau)$-IND-CPA secure
    in the random oracle model if for every $\tau$-time algorithm A, we have
    that}
    \begin{multline*}
        |\Pr[b = 1: k \leftarrow KG(),
            (m_0, m_1) \leftarrow A^{H, E^H(k,\cdot)}(),
            c \leftarrow E^H(k, m_0),
            b \leftarrow A^{H, E^H(k,\cdot)}(c)]\\ -
        \Pr[b = 1: k\leftarrow KG(),
            (m_0, m_1) \leftarrow A^{H, E^H(k,\cdot)}(),
            c \leftarrow E^H(k, m_1),
            b \leftarrow A^{H, E^H(k,\cdot)}(c)]|
        \le \epsilon
    \end{multline*}
    \textit{under the condition that $|m_0| = |m_1|$.}

    \textit{Addendum.} We considered writing the definition (inspired from
    asymmetric definitions) as something like
    \begin{multline*}
        \Pr[D^H(k, c) = m_i' :
        H \rgets \mathrm{Fun}_{M\rightarrow N},
        k \leftarrow KG(1^\eta),\\
        m_1, m_2, \dots, m_x \leftarrow A^{H, E^H(k,\cdot)}(),
        c \leftarrow E^H\left(k, m_{i \rgets \{1, \dots,x\}}\right),
        m_i' \leftarrow A^{H, E^H(k,\cdot)}(c)] \le \mu(\eta)
    \end{multline*}
    which means that the adversary attempts to guess which of $x$ different
    (adversary generated) messages the encryption oracle encrypted. However, we
    chose not to formalise this definition, as it brings decryption into play,
    which is not part of the original IND-CPA definition, and may introduce
    additional considerations.
\end{solution}

\begin{problem}
    Below are several variations of the quadratic residue zero-knowledge proof.
    Each of them is broken: either it is not a proof (lacks
    soundness\footnote{We count $1/2$-soundness as OK. But something like
    $1$-soundness would not be.}), or it is not complete (does not succeed even
    if $P$ and $V$ are honest), or it is not zero-knowledge (leaks something
    about the witness). In each case, say which property is missing, and why
    (e.g., how to attack, how to compute the witness, in which case the
    protocol does not give the right output, etc.)
    
    \underline{Original zero knowledge protocol} (the one that has
    completeness, $1/2$-soundness, and zero-knowledge):
    \begin{itemize}
        \item Input of $P$ and $V$: $N,a\in \Z_N^{*}$. From now on, everything
        happens in $\Z_N$. (Side note: $N$ here should be a composite number
        where the factorization of $N$ is not known, because if it is prime or
        if one knows factorization of $N$ then it is easy to decide whether an
        element is a quadratic residue or not. However, one should not pay much
        attention to this as it is not that important when trying to understand
        the zero-knowledge proof, it is merely a requisite that has to be
        satisfied for our setting to make sense.)
        \item Input of $P$: $b\in \Z_N$ such that $b^2=a$.
        \item $P$ picks random $d\rgets \Z_q$. $P$ computes $c\gets d^2$. $P$
        sends $c$ to $V$.
        \item $V$ picks $i\in\{0,1\}$ uniformly and sends $i$ to $P$.
        \item If $i=0$, $P$ sets $z\gets d$ and sends $z$ to $V$. If $i=1$, $P$
        sets $z\gets b\cdot d$ and sends $z$ to $V$.
        \item  If $i=0$, $V$ checks whether $z^2\iseq c$. If yes, then he
        outputs $1$, otherwise he outputs $0$.  If $i=1$, $V$ checks whether
        $z^2\iseq a\cdot c$. If yes, then he outputs $1$, otherwise he outputs
        $0$.
    \end{itemize}
    
    Here are the faulty variations:
    \begin{enumerate}
        \item \underline{Protocol 1}:
        \begin{itemize}
            \item Input of $P$ and $V$: $N,a\in \Z_N^{*}$.
            \item Input of $P$: $b\in \Z_N$ such that $b^2=a$.
            \item  $P$ picks random $d\rgets \Z_q$. $P$ computes $c\gets d^2$.
            $P$ sends $c$ to $V$.
            \item $V$ picks $i\in\{0,1\}$ uniformly and sends $i$ to $P$.
            \item If $i=0$, $P$ sets $z\gets d$ and sends $z$ to $V$. If $i=1$,
            \pmb{$P$ sets $z\gets  b+d$} and sends $z$ to $V$.
            \item If $i=0$, $V$ checks whether $z^2\iseq c$. If yes, then he
            outputs $1$, otherwise he outputs $0$. If $i=1$, $V$ checks whether
            $z^2\iseq a\cdot c$. If yes, then he outputs $1$, otherwise he
            outputs $0$.
        \end{itemize}
        
        What property does this protocol not satisfy?
        
        \item \underline{Protocol 2}:
        \begin{itemize}
            \item Input of $P$ and $V$: $N,a\in \Z_N^{*}$, \pmb{current date}.
            \item Input of $P$: $b\in \Z_N$ such that $b^2=a$.
            \item $P$ picks random $d\rgets \Z_q$. $P$ computes $c\gets d^2$.
            $P$ sends $c$ to $V$.
            \item \pmb{$V$ considers the current date, such as May the $6$th,
            for example.}\\
            \pmb{If the number of the day is even (such as May $6$th), then $V$
            sets $i$ to be $0$.}\\
            \pmb{If the number of the day is odd (such as May $7$th), then $V$
            sets $i$ to be $1$.}\\
            $V$ sends $i$ to $P$.
            \item If $i=0$, $P$ sets $z\gets d$ and sends $z$ to $V$. If $i=1$,
            $P$ sets $z\gets b\cdot d$ and sends $z$ to $V$.
            \item  If $i=0$, $V$ checks whether $z^2\iseq c$. If yes, then he
            outputs $1$, otherwise he outputs $0$. If $i=1$, $V$ checks whether
            $z^2\iseq a\cdot c$. If yes, then he outputs $1$, otherwise he
            outputs $0$.
        \end{itemize}
        
        What property does this protocol not satisfy?
        
        \item \underline{Protocol 3}:
        \begin{itemize}
            \item Input of $P$ and $V$: $N,a\in \Z_N^{*}$.
            \item Input of $P$: $b\in \Z_N$ such that $b^2=a$.
            \item $P$ picks random $d\rgets \Z_q$. $P$ computes $c\gets d^2$.
            $P$ sends $c$ to $V$.
            \item $V$ picks $i\in\{0,1\}$ uniformly and sends $i$ to $P$.
            \item \pmb{$P$ sets $z_1\gets d$ and sends $z_1$ to $V$. $P$ sets
            $z_2\gets b\cdot d$ and sends $z$ to $V$.}
            \item If $i=0$, $V$ checks whether \pmb{$z_1^2\iseq c$}. If yes,
            then he outputs $1$, otherwise he outputs $0$. If $i=1$, $V$ checks
            whether \pmb{$z_2^2\iseq a\cdot c$}. If yes, then he outputs $1$,
            otherwise he outputs $0$.
        \end{itemize}
        
        What property does this protocol not satisfy?
    \end{enumerate}
\end{problem}
\begin{solution}
    \begin{enumerate}
        \item \underline{Protocol 1}
        
        The protocol does not satisfy completeness.

        We consider the two cases:
        \begin{enumerate}
            \item If $i = 0$, $V$ receives $z$ from $P$, with $z = d$, and $V$
            can verify that $z^2 \equiv d^2\equiv c \pmod{N}$. This only shows
            that $P$ computes quadratic residues modulo $N$ correctly, and
            gives no information about whether $a$ is a quadratic residue or
            not.

            \item If $i = 1$, $V$ receives $z \equiv b + d \pmod{N}$ from $P$.
            Since $b^2 \equiv a \pmod{N}$, and $d^2\equiv c \pmod{N}$, we have
            $$
            z^2 \equiv (b + d)^2 \equiv b^2 + 2bd + d^2 \equiv
            a + 2bd + c \pmod{N}.
            $$

            Because $V$ does not know $b$ (or else there would be no need to
            prove that $a$ is a quadratic residue), nor does it know $d$, it
            cannot independently compute the value $a + 2bd + c$ to verify the
            relation.
            As such, $V$ cannot verify that $a$ is indeed a quadratic residue
            (in which case the relation would hold).
        \end{enumerate}

        We see that $V$ gets no guarantees about $a$ being a quadratic residue
        in either case, even though $a$ is in fact a quadratic residue. As
        such, the proof lacks completeness.

        \item \underline{Protocol 2}
        
        The protocol does not satisfy soundness.

        We consider a situation where $a$ is not a quadratic residue and $P$ is
        dishonest. We show that $P$ can convince (a gullible) $V$ that $a$ is a
        quadratic residue even when this is not true.

        The interaction goes in one of two ways:
        \begin{enumerate}
            \item If the number of the day is even, then $P$ knows in advance
            that $i=0$, correctly computes $c\leftarrow d^2$, and returns $c$
            to $V$. Then, when $V$ sends $i = 0$ to $P$, $P$ returns $z=d$ to
            $V$.

            $V$ can then verify that $z^2 \equiv d^2 \equiv c \pmod{N}$, i.e.,
            that $P$ computes quadratic residues correctly, and will output
            $1$.

            \item If the number of days is odd, then $P$ knows in advance that
            $i = 1$. Let us assume that $a$ is not a quadratic residue, i.e.,
            $\nexists b\in \Z_N$ s.t. $b^2\equiv a\pmod{N}$.
            
            $P$ can then behave dishonestly and pick $c$ such that $d^2 \cdot
            a^{-1}\equiv c\pmod{N}$, assuming that $\gcd{(a, N)} = 1$. $P$ then
            returns $c$
            to $V$. When $V$ sends $i=1$ to $P$, $P$ returns
            $z\equiv d\pmod{N}$.
            
            But then, for $V$, we have $z^2 \equiv d^2 \equiv d^2 \cdot
            \left(a^{-1}\cdot a\right) \equiv \left(d^2 \cdot a^{-1} \right)
            \cdot a \equiv c \cdot a\pmod{N}$ because $d^2\cdot a^{-1} \equiv c
            \pmod{N}$. Therefore, $V$ has verified the relation, and will
            output $1$, even though $a$ is not a quadratic residue.

            $V$ believes that $c$ is a quadratic residue (from (a)). $V$ also
            (wrongly) believes that $b^2 = a$, which is why $V$ does not
            suspect the attack. Hence, verifying $z^2= a\cdot c$ seems like a
            valid proof to $V$.
        \end{enumerate}

        Since a dishonest $P$ can, without detection, always make $V$ output
        $1$ (i.e., verify the proof) for some\footnote{In our attack, we
        require that $\gcd{(a, N)} = 1$} quadratic non-residues $a$, the proof
        lacks soundness.

        \item \underline{Protocol 3}
        
        The protocol does not satisfy zero-knowledge.

        $P$ operates the same way regardless of the $i$ value it receives from
        $V$. That is, $P$ invariably sends to $V$:
        \begin{itemize}
            \item $z_1 \equiv d \pmod{N}$,
            \item $z_2 \equiv b\cdot d \pmod{N}$.
        \end{itemize}

        But also, $d^2\equiv c \pmod{N}$ and $b^2\equiv a\pmod{N}$. Then, $V$
        can verify that
        $$
            z_2^2 \equiv (b\cdot d)^2 \equiv b^2d^2 \equiv ac \pmod{N}
        $$
        holds, and confirm that $a$ is indeed a quadratic residue. However,
        because $N$ is public, $V$ might\footnote{In our attack, we require
        that $\gcd{(d, N)} = 1$} be able to compute $y\equiv z_1^{-1} \equiv
        d^{-1} \pmod{N}$, and recover $b$ with $z_2y\equiv b\cdot d\cdot d^{-1}
        \equiv b\pmod{N}$.

        We see that the witness $b$ may get leaked to the verifier, and so the
        proof clearly does not satisfy the zero-knowledge property.
    \end{enumerate}
\end{solution}
\end{document}
