\documentclass{../homework}

\usepackage{enumitem}

\newtheorem{problem}{Problem}
\newenvironment{solution}[1][\it{Solution}]{\textbf{#1. } }

\newcommand{\Prx}[1]{\Pr\left[#1\right]}
\input{../name.tex}

\begin{document}

\homework{Homework \#1}
{Due: 04.03.23}{\profname}{\myname}

\begin{problem}
    Show that the equation $12x+9y = 1$ has no solutions over the integers, that
    is, there do not exist two such integers $x, y$ such that $12x + 9y = 1$.
\end{problem}
\begin{solution}
Consider $x,y\in\Z$. We are given
\begin{alignat*}{2}
                    \quad&& 12x + 9y     &= 1\\
    \Leftrightarrow \quad&& 3(4x + 3y)   &= 1\\
    \Leftrightarrow \quad&& 4x + 3y      &= \frac{1}{3}.
\end{alignat*}
Because addition and multiplication are closed binary operations on $\Z$, we
have $x,y\in\Z\Rightarrow4x,3y\in\Z\Rightarrow4x+3y\in\Z$. However,
$4x+3y=\frac{1}{3}\notin\Z$, which is a contradiction. Therefore, there do not
exist two integers $x, y$ such that $12x+9y=1$. $\square$
\end{solution}

\begin{problem}
\hfill
\begin{enumerate}[label=(\alph*)]
    \item Find a nonzero small integer $a$ such that $7^a\equiv 1 \pmod{25}$.
    \item What is $7^{123456789}\bmod{25}$?
\end{enumerate}
\end{problem}
\begin{solution}
\begin{enumerate}[label=(\alph*)]
    \item The trivial approach is to increment $a$ one at a time, and evaluate
    the result.
    
    We have:
    \begin{alignat*}{2}
        a=1:\quad&& 7^1 \bmod 25 &= 7\\
        a=2:\quad&& 7^2 \bmod 25 &= 49 \bmod 25 = 24\\
        a=3:\quad&& 7^3 \bmod 25 &= 343 \bmod 25 = (325 + 18) \bmod 25 = 18\\
        a=4:\quad&& 7^4 \bmod 25 &= \left(7^2\right)^2 \bmod 25\\
        \quad&& &=\left(7^2 \bmod{25}\right)^2 \bmod{25}\\
        \quad&& &=24^2 \bmod 25 = 576 \bmod 25 = (575 + 1) \bmod 25 = 1
    \end{alignat*}
    We see that $a = 4$ is a nonzero small integer s.t. $7^a\equiv1\pmod{25}$.
    \item Consider $a = 4b$. We have
    $7^a \bmod{25} = \left(7^4\right)^b \bmod{25} = \left(7^4 \bmod{25}\right)^b
    \bmod{25}=(1)^b\bmod{25}=1$, and so $4|a \Rightarrow 7^a\equiv 1 \pmod{25}$.

    We know that $4|100$, and so $4$ divides any decimal integer whose last two
    digits are divisible by $4$. Therefore, $4|123456788$ and we have
    $7^{123456789}\equiv1\cdot7^1\equiv7\pmod{25}$.
\end{enumerate}
\end{solution}

\begin{problem}
    Alice says: ``Think of any integer. Double it. Take the result to the fourth
    power. Add two. Square the result. Take the last digit. It's 4, isn't it?''
    Show that Alice is correct.
\end{problem}
\begin{solution}
Let $n\in\Z$. By following Alice's instructions, we get
$\left((2n)^4+2\right)^2$. To show that the last digit is $4$, we can show that
$\left((2n)^4+2\right)^2\equiv4\pmod{10}$.
\begin{alignat*}{2}
    \quad&& \left((2n)^4+2\right)^2 &\equiv 4\pmod{10}\\
    \Leftrightarrow \quad&& (2n)^8+2^2(2n)^4+4 &\equiv 4\pmod{10}\\
    \Leftrightarrow \quad&& 2^6\left(2^2n^8+n^4\right) &\equiv 0\pmod{10}\\
    \Leftrightarrow \quad&& \left(2^5n^4\right)\left(2^2n^4+1\right) &\equiv 0
    \pmod{5}
\end{alignat*}
We have
\begin{alignat*}{2}
    n\equiv0\pmod{5}\quad \lor\quad &&2^2n^4+1 &\equiv 0\pmod{5}\\
    \Leftrightarrow&& 4n^4 &\equiv -1 \pmod{5}\\
    \Leftrightarrow&& n^4 &\equiv 1 \pmod{5}\\
    \Leftrightarrow&& n^{5-1} &\equiv 1 \pmod{5}
\end{alignat*}
By Fermat's little theorem we know that $n^{5-1} \equiv 1 \pmod{5}$ holds for
all $n$ that are not multiples of $5$. But we also have that
$n\equiv 0\pmod{5}$ when $5|n$. We hence have that
$$
    \forall n\in\Z:\left(2^5n^4\right)\left(2^2n^4+1\right) \equiv 0\pmod{5},
$$
and so, by equivalence, $\left((2n)^4+2\right)^2\equiv4\pmod{10}$ for any
integer $n$. $\square$
\end{solution}

\begin{problem}
    The disease XXX has a prevalence rate of $0.001$ among the population, i.e,
    the probability that a random person has XXX is $0.001$. A test exists with
    the following properties: the probability that you will test positive if you
    have XXX, is $0.95$. The probability that you will test positive if you do
    not have XXX is $0.1$.
\begin{enumerate}[label=(\alph*)]
    \item What is the probability that a random person would test positive with
    this test?
    \item Bob does a test and gets a positive result. Bob is very worried and
    thinks he has 95\% chance of having XXX. Is Bob's concern justified? What is
    the actual probability of Bob having XXX?
    \item Bob talks to Alice about this, who calms him down somewhat. Bob now
    does the test two more times and they also come back positive. What is now
    the probability that Bob has XXX? If you think that it is not possible to
    give an exact answer to this question, explain why.
\end{enumerate}
\end{problem}
\begin{solution}
    Let us denote the events
\begin{itemize}
    \item $X:$ someone has XXX,
    \item $P:$ someone tests positive for XXX.
\end{itemize}
We have the following probabilities:
\begin{itemize}
    \item $\Prx{X}=0.001, \Prx{\widebar{X}} = 1 - \Prx{X} = 0.999$
    \item $\Prx{P|X}=0.95, \Prx{\widebar{P}|X} = 1 - \Prx{P|X} = 0.05$
    \item $\Prx{P|\widebar{X}}=0.1, \Prx{\widebar{P}|\widebar{X}}=0.9$
\end{itemize}
We will calculate other probabilities as needed.
\begin{enumerate}[label=(\alph*)]
    \item By the law of total probability, we have
    $\Prx{P}=\Prx{P|X}\Prx{X} + \Prx{P|\widebar{X}}\Prx{\widebar{X}}=
    0.95\cdot0.001 + 0.1\cdot0.999=0.10085$, which is the probability of a
    random person testing positive with this test.

    \item Bob's concern is not justified. $95\%$ is the probability that the
    test is positive \textit{if} Bob has XXX. However, the test may be positive
    without Bob having XXX. Bob hence seeks $\Prx{X|P}$, which is the
    probability of having XXX if the test is positive.
    
    We can calculate this using Bayes' theorem:
    $$
        \Prx{X|P}=\frac{\Prx{P|X}\Prx{X}}{\Prx{P}}=
        \frac{0.95\cdot0.001}{0.10085}
        \approx 0.00942.
    $$
    The probability that Bob has XXX based on a single test is $0.94\%$.

    \item Let us assume that Bob has XXX. The probability that he tests positive
    is 0.95. Because testing is an independent event (does not depend on
    previous test results), the probability that Bob tests positive thrice when
    having XXX is $\left( \Prx{P|X} \right)^3 = 0.95^3$. We write it as
    $\Prx{P^3|X}=0.95^3$.

    By using Bayes' theorem, we have
    $$
        \Prx{X|P^3}=\frac{\Prx{P^3|X}\Prx{X}}{\Prx{P^3}}=
        \frac{\Prx{P^3|X}\Prx{X}}
        {\Prx{P^3|X}\Prx{X}+\Prx{P^3|\widebar{X}}\Prx{\widebar{X}}},
    $$
    and we also have $\Prx{P^3|\widebar{X}}
    = \left(\Prx{P|\widebar{X}}\right)^3
    = 0.1^3$, and $\Prx{X}=0.001=0.1^3$.

    We can then calculate
    $$
        \Prx{X|P^3}=
        \frac{0.95^3\cdot 0.1^3}{0.95^3\cdot 0.1^3+0.1^3\cdot 0.999}=
        \frac{0.95^3\cdot 0.1^3}{0.1^3(0.95^3 + 0.999)}=
        \frac{0.95^3}{0.95^3+0.999}\approx 0.4619.
    $$
    The probability that Bob has XXX considering he tested positive thrice is
    $46.19\%$.

    This probability is of course based on the assumption that the test outcome
    is only affected by the false positive/false negative rate and whether Bob
    has XXX. It is possible that there are some unknown factors that skew the
    test in Bob's case, which would make the resulting probability inaccurate.

    \iffalse
    \item The test being positive does not make it more likely for the next test
    to be positive, even if we perceive it as such. What really makes the test
    more likely to be positive is if Bob has the disease. That is, the test
    result is independent of previous tests, and is dependent only on Bob having
    XXX or not. \textit{(at least it seems reasonable for me to assume so)}

    Let $P_i$ be the event that test $i$ is positive. We have
    $$
        \Pr(P_{i+1}|P_i)=\frac{\Pr(P_{i+1}\cap P_i)}{\Pr(P_i)} =
        \frac{
            \Pr(P_{i+1}\cap P_i|X)\Pr(X) +
            \Pr(P_{i+1}\cap P_i|\widebar{X})\Pr(\widebar{X})
        }
        {\Pr(P_i|X)\Pr(X) + \Pr(P_i|\widebar{X})\Pr(\widebar{X})}
    $$
    We must now figure out the value of $\Pr(P_{i+1}\cap P_i|X)$ and
    $\Pr(P_{i+1}\cap P_i|\widebar{X})$.

    If we fix that Bob has (or does not have) the disease, the test result is
    still independent of previous results, as testing is an independent event.
    As such, we have
    $$
        \Pr(P_{i+1}\cap P_i|X) = \Pr(P_{i+1}|X)\Pr(P_i|X) =
        \left(\Pr(P|X)\right)^2
    $$
    and $\Pr\left(P_{i+1}\cap P_i|\widebar{X}\right)=
    \left(\Pr\left(P|\widebar{X}\right)\right)^2$
    following the same reasoning.
    \fi
\end{enumerate}
\end{solution}

\begin{problem}
    The following is a very simplified version of an existing post-quantum
    cryptosystem (i.e a cryptosystem that, as far as we know, large quantum
    computers cannot break).
    
    There are some fixed parameters $t, s$ and $p$. A message must be in
    $[0, p - 1]$. When we encrypt a message $m$, we sample an integer $e$
    uniformly from $[-t, t]$. This is called a noise term. We will not specify
    how encryption is done.
    
    Decrypting has two steps. First step, which we will not specify, will
    return the term $m + p\cdot e =: m'$. After that, we will compute
    $m'\pmod{p}$ and obtain the original message $m$. This scheme also has the
    added property that, under some constraints, it is possible to add and
    multiply ciphertexts. Thus, it is possible to perform operations on, say
    $Enc(a)$ and $Enc(b)$, to obtain $Enc(a + b)$ and $Enc(a\cdot b)$ without
    knowing the decryption key nor $a$ nor $b$.
    
    Thus, for addition, if we encrypt a message $m_1$ with noise term $e_1$ and
    message $m_2$ with noise term $e_2$, then after performing the first step of
    decryption, we obtain $m_1 + m_2 + p(e_1 + e_2)$. We take the modulo $p$ and
    have as the result $m_1 + m_2 \pmod{p}$. (Something similar happens with
    multiplication, but we will not talk about it here)
    
    However, these operations can fail and give us incorrect results when we are
    unlucky. Namely, for addition, if it happens that
    $e_1 + e_2 > s$ or $e_1 + e_2 < -s$, then the decryption fails.
    \begin{enumerate}[label=(\alph*)]
        \item Let $t = 1$, i.e the noise is picked from $[-1, 0, 1]$. Let
        $s = 1$. Two messages are encrypted and their encryptions are added.
        What is the probability that the decryption fails, that is,
        $e_1 + e_2 > 1$ or $e_1 + e_2 < -1$?
        \item Let $t > 1$, so the noise is picked from
        $[-t,\dots, -1, 0, 1,\dots, t]$. Let $s = t$. Two messages are encrypted
        and their encryptions are added. What is the probability that the
        decryption fails? (Hint: The formula for the sum of first $k$ numbers is
        $1 + 2 +\dots+ k = \frac{k(k+1)}{2}$.)
    \end{enumerate}
\end{problem}
\begin{solution}
\begin{enumerate}[label=(\alph*)]
%    \item We have $e_1,e_2\in\{-1, 0, 1\}$. We can have $e_1=e_2$, and we do not
%    consider the order because addition is commutative. We therefore have a case
%    of choosing $2$-combinations with repetition from $3$ elements, and we have
%    a total of $\binom{3+2-1}{2}=6$ such combinations for $e_1,e_2$.
%    
%    Decryption can however only fail in two cases: if
%    $e_1 = 1 = e_2$ or $e_1 = -1 = e_2$. Therefore the probability of failure is
%    $\frac{2}{6}=\frac{1}{3}=0.\widebar{3}$.

    \item We have $e_1,e_2\in\{-1, 0, 1\}$. We therefore have $3^2=9$ total
    ordered pairs, because $e_1,e_2$ can both take $3$ values, neither depends
    on the other, and we can have $e_1=e_2$.

    Decryption can however only fail in two cases: if
    $e_1 = 1 = e_2$ or $e_1 = -1 = e_2$. Therefore the probability of failure is
    $\frac{2}{9}=0.\widebar{2}$.


    \item There are $2t + 1$ elements in the set $\{-t,\dots,-1,0,1,\dots,t\}$,
    and so there are now a total of $(2t+1)^2$ ordered pairs of elements.
%    $$
%    \binom{(2t + 1) + 2 - 1}{2}=\frac{(2t +2)!}{2(2t)!}=
%    \frac{(2t+2)(2t+1)(2t)!}{2(2t)!}=\frac{(2t+2)(2t+1)}{2}=(t+1)(2t+1)
%    $$
%    combinations.
    
    Let us restrict ourselves to when $e_1, e_2$ are positive, that is,
    $e_1, e_2\in\{1,2,\dots,t\}$. We do not need to consider cases where either
    $e$ is $0$, as then it is not possible to have $e_1+e_2>t$. How many
    calculations can we make so that $e_1+e_2>t$?

    We have that for $e_1 = t - x$, with $x\in\{1,2,\dots,t-1\}$, then $x < e_2
    \le t$:
    \begin{itemize}
        \item For $e_1=t$, we have $e_1+e_2>t$ for any $e_2\in\{1,2,\dots,t\}$,
        so there are $t$ possibilities.
        \item For $e_1=t-1$, $e_1+e_2>t$ for $e_2\in\{2,3,\dots,t\}$, and there
        are $t-1$ possibilities.
        
        $\vdots$
        \item For $e_1=1$, $e_1+e_2>t$ there is only one possibility, which is
        $e_2 = t$.
    \end{itemize}
    We see then that the number of possibilities forms a simple additive
    sequence, and we can sum up the possibilities for each case to get the total
    number of possibilities that result in $e_1+e_2>t$. We have
    $$
        1 + 2 + \dots + (t-1) + t = \frac{t(t+1)}{2}
    $$
    total possibilities. We must also consider the negative cases, so by
    symmetry around $0$, we multiply the number of cases by $2$. We do not have
    to consider cases where either $e$ is $0$. There are then $t(t+1)$ cases
    that satisfy $\lvert e_1 + e_2 \rvert > t$.

    To get the probability of decryption failure, we simply divide the number of
    possible failure cases by the number of total possible cases. The
    probability that decryption fails for $s=t$ is hence
    $$
        \frac{t(t+1)}{(2t+1)^2}.
    $$

    To go a step further, we have that 
    $$
        \lim_{t\rightarrow\infty} \frac{t(t+1)}{(2t+1)^2} =
        \lim_{t\rightarrow\infty} \frac{t^2+t}{4t^2+4t+1} =
        \frac{1}{4} \lim_{t\rightarrow\infty}
        \frac{t^2\left(1+\frac{1}{t}\right)}
        {t^2\left(1+\frac{1}{t}+\frac{1}{4t^2}\right)} = \frac{1}{4}
    $$
    and so, as $t$ grows larger, the probability of failure when $t=s$ nears
    $25\%$.

    \iffalse
    We can sum up the number of possibilities to get $\frac{t(t+1)}{2}$, but
    this represents the number of ordered pairs $(e_1, e_2)$. Since we do not
    consider order for the sum, because addition is commutative, we must get rid
    of the pairs producing duplicate sums.

    We can intuitively see that we have duplicates for $0< e_1 \le t/2$, because
    $$
        0 < e_1 \le t/2\quad\land\quad e_1+e_2 >t\quad\Rightarrow\quad e_2 > t/2
    $$
    and for every $x\in\{\lceil t/2 \rceil,\dots,t-1\}$ we have already
    accounted for the unordered pairs $\{t-x, y\}$, where $x < y \le t$.%\in\{t/2,\dots,t\}$.
    
    We cannot simply half the number of possibilities however, since the special
    case where $e_1=e_2$ satisfies $e_1+e_2>t$ is only true when
    $e_1=e_2 > \lfloor t/2 \rfloor$. When $t$ is even, there are $t/2$ such
    special cases. When $t$ is odd, there are $(t+1)/2$ such cases, because
    $e_1 = e_2 > \lfloor t/2 \rfloor \Leftrightarrow
    e_1 = e_2 \ge \lceil t/2 \rceil$ for odd $t$.
    
    because the
    ``middle'' special case, e.g., $\{3, 3\}$ for $t=5$, also satisfies
    $e_1+e_2>t$, and so we must include it. Another way to think about it is
    that we additionally consider the case $\{0,0\}$ to make the number of cases
    even, and hence $(t+1)/2$ and not $(t-1)/2$.
    
    Therefore, we must first subtract the number of special cases, before we can
    half the number of possibilities. We then add back the number of special
    cases after halving.

    If $t$ is even:

    $$
        \frac{\frac{t(t+1)}{2}-\frac{t}{2}}{2}+\frac{t}{2} =
        \frac{\frac{t(t+1)}{2}-\frac{t}{2}+\frac{2t}{2}}{2} =
        \frac{t(t+1) + t}{4} =
        \frac{t(t+2)}{4}
    $$
    
    If $t$ is odd:

    $$
        \frac{\frac{t(t+1)}{2}-\frac{t+1}{2}}{2}+\frac{t+1}{2} =
        \frac{\frac{t(t+1)-(t+1)}{2}+\frac{2(t+1)}{2}}{2} =
        \frac{t(t+1)+(t+1)}{4} = \frac{(t+1)^2}{4}
    $$

    By symmetry around $0$, to consider also the negative cases, so we multiply
    the number of cases by $2$. We hence have a total
    of
    $$
        \frac{t(t+2)}{2} \text{ cases for even $t$, }
        \frac{(t+1)^2}{2} \text{ cases for odd $t$}
    $$
    that satisfy $e_1+e_2>t$.
    \fi
\end{enumerate}
\end{solution}

\end{document}