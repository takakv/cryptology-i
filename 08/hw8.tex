\documentclass{../homework}

\usepackage{etoolbox}
\usepackage{xcolor}
\hypersetup{
    colorlinks,
    linkcolor={red!50!black},
    citecolor={blue!50!black},
    urlcolor={blue!80!black}
}

\newtheorem{problem}{Problem}
\newenvironment{solution}[1][\it{Solution}]{\textbf{#1. } }

\input{../name.tex}

\newcommand{\rgets}{\stackrel{\mathdollar}\leftarrow}
\newcommand{\iseq}{\stackrel{?}=}

\newcommand{\Enc}{\text{Enc}}
\newcommand{\Dec}{\text{Dec}}
\newcommand{\tmac}{\text{TreeMAC}}

\begin{document}

\homework{Homework \#8}
{Due: 26.05.23}{\profname}{\myname}

\begin{problem}
    Your small organization of $21$ people has decided to elect a new leader.
    There are two candidates for the leader, you and Alice.
    Because your organization is a forward-looking one, you have decided to do
    digital voting using secure multiparty computation instead of paper-voting.
    You are using a fully homomorphic homomorphic IND-CPA public-key encryption
    scheme, namely, an encryption scheme with the property that there are
    easy-to-compute operations $\oplus$ and $\otimes$ such that for all
    plaintexts $m_1,m_2$ it holds that
    $$
        \Enc(pk,m_1)\oplus \Enc(pk,m_2)=Enc(pk,m_1+m_2)
    $$
    and
    $$
        \Enc(pk,m_1)\otimes \Enc(pk,m_2)=\Enc(pk,m_1\cdot m_2).
    $$
    
    There is also an easy-to-compute operation $\ominus$ that allows you to
    compute
    $$
        \Enc(pk,m_1)\ominus \Enc(pk,m_2)=\Enc(pk,m_1- m_2)
    $$
    for any $m_1,m_2$.
    You can assume that the messages $m_i$ are all in some kind of a large
    field $\Z_q$.
    
    You can assume that this encryption scheme has a property called
    \emph{circuit privacy} which roughly means that if one is given a secret
    key and $\Enc(x)$, one cannot tell how this $x$ was computed.
    One might imagine that, given a fully homomorphic encryption scheme, one
    can obtain
    $\Enc(5)=\Enc(2)\oplus \Enc(3)$
    or
    $\Enc(5)=\Enc(1)\oplus \Enc(4)$.
    Given a secret key, it might be possible to distinguish between these two
    cases.
    Circuit privacy says that even if you have the secret key, it is not
    possible to distinguish this --- for any two
    $f_1(x_1,\dots,x_k)=c$
    and
    $f_2(y_1,\dots,y_m)=c$,
    it is not possible to tell whether $\Enc(c)$ was obtained by starting with
    $\Enc_{pk}(x_1),\dots,\Enc_{pk}(x_k)$
    and homomorphically applying the function $f_1$ or by starting with
    $\Enc_{pk}(y_1),\dots,\Enc_{pk}(y_m)$
    and homomorphically applying the function $f_2$.
    (Previously, we haven't had any security properties for the case when the
    attacker knows the secret key. In fact, fully homomorphic lattice
    encryption schemes do not usually have circuit privacy, some more
    operations are necessary for that.)
% $\Enc(9)=\Enc(1)\oplus(\Enc())

 %(e.g $Enc(pk,10)+Enc(pk,2)=Enc(pk,12)$, $Enc(pk,3)+Enc(pk,3)=Enc(pk,6)$ )
 
    The voting scheme is the following.
    There are two somewhat trusted parties, Bob and Charlie.
    It is assumed that Bob and Charlie will not collude because they hate each
    other.
    Also we will assume that they will follow the protocol and cannot be bribed
    or otherwise influenced.
    
    Charlie will generate a pair of keys $(pk,sk)$ and give everyone the public
    key $pk$. Now, if a voting  party $P_i$ wishes to vote for you, they will
    compute an encryption of $1$, if they want to vote for Alice, they will
    compute an encryption of $0$. That is, $c_i=Enc(pk,1)$ if $P_i$ wants to
    vote for you and  $c_i=Enc(pk,0)$ if $P_i$ wants to vote for Alice. All the
    voting parties will send their ciphertexts $c_i$ to Bob, who will compute
    $c:=c_1\oplus c_2 \oplus \dots \oplus c_{21}$. Bob sends $c$ to Charlie who
    will decrypt $c$ with his secret key and produce $m:=Dec(sk,c)$. Charlie
    will tell $m$ to everybody. If $m\in[0,10]$, Alice will be the winner and
    if $m\in [11,21]$, you will be the winner.
    
    \begin{enumerate}
        \item Let's assume you want to cheat to guarantee that you will be the
        leader. From chat around the office you know that at least $3$ other
        people (i.e not including yourself) will vote for you but no more than
        $6$ other people will. What should you do to guarantee that you win and
        that people don't notice that somebody cheated? You can assume that all
        the other participants (there's $20$ of them) behave honestly, all of
        them vote and that all of them vote for either Alice or you. (For the
        sake of clarity - you also have a vote.) You will not be able to post a
        vote on someone else's behalf or change other people's votes. Otherwise
        you are allowed to not follow the protocol. What should you do?
        
        \item For extra security, Bob will perform the additional following
        operation for each $i$. Besides doing what is described in the main
        description of the protocol, the following will be also done. Bob will,
        for each $i$ take $c_i$, compute $v_i=\Enc_{pk}(1)$, and compute
        $d_i:=(v_i\ominus c_i)\otimes c_i$. He will then send all the $d_i$ to
        Charlie, along with the name of the person who submitted the
        corresponding $c_i$. Charlie will then decrypt the $d_i$ and send them
        to everyone, including which is name associated with which $d_i$.
        Assume Bob and Charlie are honest. How does this extra operation
        prevent attacks? Does this thing still preserve the privacy of the
        vote?
    \end{enumerate}
\end{problem}
\begin{solution}
    \begin{enumerate}
    \item Assuming that my intel holds, I know I will receive at least $3$
    votes.
    Since I need $\ge 11$ votes to win, I need to affect the result by at most
    $11-3 = 8$ votes.
    If I can create $8$ votes out of thin air, and all $6$ friendly colleagues
    vote for me, then the total vote count for me is $14\in[11, 21]$.
    Clearly I will also win, but with a number of votes that remains
    mathematically possible, and plausible.

    Let us assume that I will compute an encryption of $8$.
    Because the encryption scheme has IND-CPA, and Bob does not have access to
    $sk$, he cannot detect that I have encrypted a number other than $0$ or
    $1$.
    Bob then sums up the ciphertexts and sends $c$ to Charlie.
    Because the encryption scheme has circuit privacy, and Charlie does not
    gain access to the discrete ciphertexts, he cannot detect either that a
    ciphertext included in the sum $c$ was the encryption of a value too large.

    When Charlie decrypts $c$, he will get $m\in[11, 14]$, and my win is
    guaranteed.
    Because the result is plausible, and neither Bob nor Charlie can
    mathematically detect my foul-play, I have cheated my election win without
    discovery.
    Of course, we assume that other voters do not conduct an internal
    ``exit poll'', or similar.

    \item This extra operation serves as a range proof of the vote.
    We distinguish between three cases.
    \begin{enumerate}
        \item If someone votes $0$, we have that
        $$
        d=(\Enc_{pk}(1) \ominus \Enc_{pk}(0))\otimes \Enc_{pk}(0)
        = \Enc_{pk}(0)\otimes \Enc_{pk}(0) = \Enc_{pk}(0),
        $$
        and so $\Dec_{sk}(d) = 0$.
        
        \item If someone votes $1$, we have that
        $$
        d=(\Enc_{pk}(1) \ominus \Enc_{pk}(1))\otimes \Enc_{pk}(1)
        = \Enc_{pk}(0)\otimes \Enc_{pk}(1) = \Enc_{pk}(0),
        $$
        and so $\Dec_{sk}(d) = 0$.

        \item If someone votes $x \in\Z_q\setminus\{0, 1\}$, then we have
        $\Enc_{pk}(1)\ominus \Enc_{pk}(x) = \Enc_{pk}(1-x)$. Because
        $1 + (-1) = 0$, $x \neq 1 \Rightarrow -x \neq -1$, and the additive
        inverse in a field is unique, then clearly $1 - x = 1 + (-x) \neq 0$.
        Moreover, clearly $x\neq 0$ since $x\in\Z_q\setminus\{0,1\}$.

        It follows that $(1-x)\cdot x = x-x^2 \neq 0$, and so
        $\Dec_{sk}(d)\neq 0$ with
        $$
        d=(\Enc_{pk}(1) \ominus \Enc_{pk}(x))\otimes \Enc_{pk}(x)
        = \Enc_{pk}(1-x)\otimes \Enc_{pk}(x) = \Enc_{pk}(x-x^2).
        $$
    \end{enumerate}

    We see that if someone encrypts a number other than $0$ or $1$, then the
    extra operation will expose the cheat.
    We also see that the range proof will output $0$ regardless of whether the
    vote was for $0$ or $1$, and so the secrecy of the vote is not broken.
    Therefore, when Charlie publishes the (voter, proof) pairs, everyone can
    verify if some specific voter was honest, but vote secrecy is preserved for
    honest voters.

    For cheating voters, it might be possible to solve the equation
    $x-x^2 = \Dec_{sk}(d_i)$ in $\Z_q$ and therefore recover full or partial
    information about the vote, though this does not correspond to a candidate
    in any case.
    \end{enumerate}
\end{solution}

\begin{problem}
    Which were the three most difficult homework problems this semester? Which
    were the three easiest? Approximately how much time did it take you to
    solve them?
\end{problem}
\begin{solution}

\end{solution}
\end{document}
