\documentclass{../homework}

\usepackage{etoolbox}
\usepackage{xcolor}
\hypersetup{
    colorlinks,
    linkcolor={red!50!black},
    citecolor={blue!50!black},
    urlcolor={blue!80!black}
}

\newtheorem{problem}{Problem}
\newenvironment{solution}[1][\it{Solution}]{\textbf{#1. } }

\newcommand{\Prx}[1]{\Pr\left[#1\right]}
\newcommand{\ursample}{\overset{{\scriptscriptstyle\$}}{\leftarrow}}
\newcommand{\st}{\text{ such that }}
\input{../name.tex}

\DeclareMathOperator*{\concat}{\scalerel*{\Vert}{\sum}}
\newcommand\iv{\mathit{iv}}

\begin{document}

\homework{Homework \#2}
{Due: 18.03.23}{\profname}{\myname}

\begin{problem}
    
\end{problem}
\begin{solution}
    
\end{solution}

\begin{problem}
    Consider the following mode of operation (which I call ``CBC+1''):
    
    CBC+1 is very similar to CBC, but instead of picking the $\iv$ at random, we
    increase $iv$ after each encryption by one, otherwise the scheme is the
    same. That is, at first we pick at random some initialization vector $\iv$.
    After that, $\iv$ is increased by one at every encryption.
    
    Show that this scheme does not have IND-CPA.
\end{problem}
\begin{solution}
    Let $E(k, \cdot)$ be the encryption oracle that receives a set $S$ of
    plaintexts, picks a plaintext $m_i$ uniformly at random from the set $S$,
    and returns the the encryption $c=E(k, m_i)$.

    The oracle first generates a key $k\leftarrow KG()$ and picks the $\iv$ at
    random. Then, for every encryption, the oracle re-uses the same key $k$.
    After each encryption, the oracle increments $\iv$ by one, and uses the new
    $\iv$ for the next encryption.

    Let $\mathcal{M}$ be the message space, and $m_0, m_1 \in \mathcal{M}$ two
    distinct messages picked by the attacker $A$.

    We show that for a game involving only two messages, we can win the game
    with complete certainty in only two steps.

\begin{enumerate}
    \item In the first step of the game, $A$ sends $m_0, m_1$ to the oracle
    $E(k, \cdot)$, and receives $c = E(k, m_x)$, where $x\in\{0, 1\}$.

    However, what $A$ really receives is
    $$
        c = c_0 \concat_{i=1} E(k, m_{x_i} \oplus c_{i-1}),\quad c_0 = \iv
    $$
    where $m_{x_i}$ represents the $i$-th block of $m_x$, $c_i$ represents the
    $i$-th ciphertext block, and $||$ is the concatenation operator.

    The attacker can recover the random $\iv$ from $c_0$.

    \item The attacker prepares a new plaintext based on $m_0$. Namely, the
    attacker splits $m_0$ into blocks, and replaces the first block\footnotemark
    \ $m_{0_1}$ with $m_{0_1} \oplus (\iv + 1) \oplus \iv$. The attacker then
    reassembles the message from the blocks, and obtains $m_0'$ with the
    modified first block. In effect, we have $m_0' = (m_{0_1} \oplus (\iv + 1)
    \oplus \iv) \concat_{i = 2} m_{0_i}$. Notice that $m_{0_i}' = m_{0_i}$ for
    $i > 1$.
    \footnotetext{Here we start indexing from $1$. There is no $m_{0_0}$.}

    Next, $A$ sends $m_0'$ to the oracle, and receives $c'=E(k, m_0')$. During
    encryption, the following happens:
    \begin{align*}
    c' &=
    c_0' \concat_{i=1} E(k, m'_{0_i} \oplus c_{i-1}'),\quad c_0' = \iv'=\iv+1\\
    &= c_0' || (m_{0_1}'\oplus\iv') \concat_{i=2} E(k, m_{0_i}\oplus c_{i-1}')\\
    &= c_0' ||
    \left((m_{0_1} \oplus (\iv + 1) \oplus \iv) \oplus (\iv + 1)\right)
    \concat_{i=2} E(k, m_{0_i}\oplus c_{i-1}')\\
    &= c_0' || (m_{0_1} \oplus \iv) \concat_{i=2} E(k, m_{0_i}\oplus c_{i-1}')\\
    &= c_0' \concat_{i=1} E(k, m_{0_i}\oplus c_{i-1}),\quad c_0 = \iv = c_0' - 1
    \end{align*}

    We see that aside from $c_0'$, the encryption of $m_0'$ produces the same
    ciphertext as if $m_0$ was encrypted with $iv$. As a result, the attacker
    can now verify if $c_i = c_i'$ for all $i>1$. If this holds, then the
    attacker knows that $c = E(k, m_0)$, otherwise $c = E(k, m_1)$.
\end{enumerate}

We have shown that under a chosen plaintext attack, the attacker is able to
distinguish which message was encrypted, and hence, by definition, the scheme
does not have IND-CPA.\qed

A practical example of this attack is shown in \texttt{cbc-plus-one.py}.

\textit{Additional observation:}

More generally, this attack extends to $n$ number of initial, distinct
plaintexts provided to the oracle. In such cases, the game will take at most $n$
steps to win.

We can easily convince ourselves that this is true by giving the oracle $m_n'$,
where $m_n' = (m_{n_1} \oplus (\iv + y) \oplus \iv) \concat_{i = 2} m_{n_i}$,
and $y$ is the increment number of the $\iv$ for the given encryption. After the
$n$-th step, we will have the ciphertexts for $n-1$ plaintexts, and hence can
certainly win the game by comparing the plaintext-ciphertext pairs with the
original challenge ciphertext.

$n-1$ ciphertexts are sufficient, because if there is no match by then, the
unmatched chosen plaintext will be the answer. Furthermore, the game can end
before $n$ steps, as the attacker may stop at the first matched pair.
\end{solution}

\end{document}
