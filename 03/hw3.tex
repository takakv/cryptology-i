\documentclass{../homework}

\usepackage{etoolbox}
\usepackage{xcolor}
\hypersetup{
    colorlinks,
    linkcolor={red!50!black},
    citecolor={blue!50!black},
    urlcolor={blue!80!black}
}

\newtheorem{problem}{Problem}
\newenvironment{solution}[1][\it{Solution}]{\textbf{#1. } }

\newcommand{\Prx}[1]{\Pr\left[#1\right]}
\newcommand{\ursample}{\overset{{\scriptscriptstyle\$}}{\leftarrow}}
\newcommand{\st}{\text{ such that }}
\input{../name.tex}

\DeclareMathOperator*{\concat}{\scalerel*{\Vert}{\sum}}
\newcommand\iv{\mathit{iv}}

\begin{document}

\homework{Homework \#3}
{Due: 04.04.23}{\profname}{\myname}

\begin{problem}
    In the file {\tt bad-aes.py} is a modified version of AES given in the
    function {\tt bad-aes}.
    
    In that version, the key is simply sampled randomly (not according to a
    keyschedule algorithm) and in the AES, the Mix-Column step is omitted. Your
    task is to distinguish this AES from a (lazily sampled) random permutation. 
\begin{enumerate}
    \item Write  the functions {\tt adv1} and {\tt adv2}. These
    adversary-functions are put into a bigger program {\tt guessing-game}.
    {\tt adv1} is supposed to produce a vector of plaintexts $(m_0,\dots,m_n)$
    where each plaintext $m_i$ has a length of $16$ bytes. A random coin $b$ is
    then flipped. If $b=0$, then the game lazily samples a permutation $\pi$ on
    the message space and  gives to {\tt adv2} the vector
    $(\pi(m_0),\dots,\pi(m_n))$. If $b=1$, then the game randomly samples a key
    $k$ and  gives to {\tt adv2} the vector
    $(\texttt{bad-aes}(k,m_0),\dots,\texttt{bad-aes}(k,m_n))$. The task of
    {\tt adv2} is to output a bit $b'$ such that $\Pr[b=b']>\frac{1}{2}+\delta$
    such that $\delta$ is large enough. (For example, a success probability of
    $55\%$ is good enough.)
    
    The guessing-game is run for 3000 times and if your program is successful
    more than 1650 times, you get the points. If you are not able to write the
    adversaries, at least try to explain how one could tackle this problem.
    
    Try not to make your code run for too long.

    \item The power that the adversary (that is, you) has in this game is
    slightly weaker than it has in the game of Weak Pseudo-Random Permutation.
    What capability or capabilities is this adversary missing?
\end{enumerate}
\end{problem}
\begin{solution}
\begin{enumerate}
    \item \textit{Implemented in code.} In the given implementation, when
    provided with only 0-bytes, the round key addition result will have as many
    set bytes as the key. Therefore, by setting only an additional byte in for a
    second message, the encryption result will differ by a byte only. This is
    clearly not the case for a random permutation, and allows us to win the
    game, by detecting the one byte difference.

    \item In this game, the adversary is not allowed to be adaptive. That is,
    while the adversary may send any number of queries to the oracle, these
    queries must be made at the same time. For any subsequent set of queries,
    the key, and hence the apparent permutation changes. In essence, here, the
    adversary only has one attempt at a time to craft a payload that wins them
    the game. Conversely, in a game of Weak Pseudo-Random Permutation, the
    adversary can be adaptive, allowing them to tune the payload to a response
    in an attempt to win the game.
\end{enumerate}
\end{solution}

\begin{problem}
    Consider the following mode of operation (which I call ``CBC+1''):
    
    CBC+1 is very similar to CBC, but instead of picking the $\iv$ at random, we
    increase $iv$ after each encryption by one, otherwise the scheme is the
    same. That is, at first we pick at random some initialization vector $\iv$.
    After that, $\iv$ is increased by one at every encryption.
    
    Show that this scheme does not have IND-CPA.
\end{problem}
\begin{solution}
    Let $E(k, \cdot)$ be the encryption oracle that receives a set $S$ of
    plaintexts, picks a plaintext $m_i$ uniformly at random from the set $S$,
    and returns the the encryption $c=E(k, m_i)$.

    The oracle first generates a key $k\leftarrow KG()$ and picks the $\iv$ at
    random. Then, for every encryption, the oracle re-uses the same key $k$.
    After each encryption, the oracle increments $\iv$ by one, and uses the new
    $\iv$ for the next encryption.

    Let $\mathcal{M}$ be the message space, and $m_0, m_1 \in \mathcal{M}$ two
    distinct messages picked by the attacker $A$.

    We show that for a game involving only two messages, we can win the game
    with complete certainty in only two steps.

\begin{enumerate}
    \item In the first step of the game, $A$ sends $m_0, m_1$ to the oracle
    $E(k, \cdot)$, and receives $c = E(k, m_x)$, where $x\in\{0, 1\}$.

    However, what $A$ really receives is
    $$
        c = c_0 \concat_{i=1} E(k, m_{x_i} \oplus c_{i-1}),\quad c_0 = \iv
    $$
    where $m_{x_i}$ represents the $i$-th block of $m_x$, $c_i$ represents the
    $i$-th ciphertext block, and $||$ is the concatenation operator.

    The attacker can recover the random $\iv$ from $c_0$.

    \item The attacker prepares a new plaintext based on $m_0$. Namely, the
    attacker splits $m_0$ into blocks, and replaces the first block\footnotemark
    \ $m_{0_1}$ with $m_{0_1} \oplus (\iv + 1) \oplus \iv$. The attacker then
    reassembles the message from the blocks, and obtains $m_0'$ with the
    modified first block. In effect, we have $m_0' = (m_{0_1} \oplus (\iv + 1)
    \oplus \iv) \concat_{i = 2} m_{0_i}$. Notice that $m_{0_i}' = m_{0_i}$ for
    $i > 1$.
    \footnotetext{Here we start indexing from $1$. There is no $m_{0_0}$.}

    Next, $A$ sends $m_0'$ to the oracle, and receives $c'=E(k, m_0')$. During
    encryption, the following happens:
    \begin{align*}
    c' &=
    c_0' \concat_{i=1} E(k, m'_{0_i} \oplus c_{i-1}'),\quad c_0' = \iv'=\iv+1\\
    &= c_0' || (m_{0_1}'\oplus\iv') \concat_{i=2} E(k, m_{0_i}\oplus c_{i-1}')\\
    &= c_0' ||
    \left((m_{0_1} \oplus (\iv + 1) \oplus \iv) \oplus (\iv + 1)\right)
    \concat_{i=2} E(k, m_{0_i}\oplus c_{i-1}')\\
    &= c_0' || (m_{0_1} \oplus \iv) \concat_{i=2} E(k, m_{0_i}\oplus c_{i-1}')\\
    &= c_0' \concat_{i=1} E(k, m_{0_i}\oplus c_{i-1}),\quad c_0 = \iv = c_0' - 1
    \end{align*}

    We see that aside from $c_0'$, the encryption of $m_0'$ produces the same
    ciphertext as if $m_0$ was encrypted with $iv$. As a result, the attacker
    can now verify if $c_i = c_i'$ for all $i>1$. If this holds, then the
    attacker knows that $c = E(k, m_0)$, otherwise $c = E(k, m_1)$.
\end{enumerate}

We have shown that under a chosen plaintext attack, the attacker is able to
distinguish which message was encrypted, and hence, by definition, the scheme
does not have IND-CPA.\qed

A practical example of this attack is shown in \texttt{cbc-plus-one.py}.

\textit{Additional observation:}

More generally, this attack extends to $n$ number of initial, distinct
plaintexts provided to the oracle. In such cases, the game will take at most $n$
steps to win.

We can easily convince ourselves that this is true by giving the oracle $m_n'$,
where $m_n' = (m_{n_1} \oplus (\iv + y) \oplus \iv) \concat_{i = 2} m_{n_i}$,
and $y$ is the increment number of the $\iv$ for the given encryption. After the
$n$-th step, we will have the ciphertexts for $n-1$ plaintexts, and hence can
certainly win the game by comparing the plaintext-ciphertext pairs with the
original challenge ciphertext.

$n-1$ ciphertexts are sufficient, because if there is no match by then, the
unmatched chosen plaintext will be the answer. Furthermore, the game can end
before $n$ steps, as the attacker may stop at the first matched pair.
\end{solution}

\end{document}
