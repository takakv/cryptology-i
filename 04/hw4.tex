\documentclass{../homework}

\usepackage{etoolbox}
\usepackage{xcolor}
\hypersetup{
    colorlinks,
    linkcolor={red!50!black},
    citecolor={blue!50!black},
    urlcolor={blue!80!black}
}

\newtheorem{problem}{Problem}
\newenvironment{solution}[1][\it{Solution}]{\textbf{#1. } }

\newcommand{\Prx}[1]{\Pr\left[#1\right]}
\newcommand{\ursample}{\overset{{\scriptscriptstyle\$}}{\leftarrow}}
\newcommand{\st}{\text{ such that }}
\input{../name.tex}

\DeclareMathOperator*{\concat}{\scalerel*{\Vert}{\sum}}
\newcommand\iv{\mathit{iv}}

\begin{document}

\homework{Homework \#4}
{Due: 15.04.23}{\profname}{\myname}

\begin{problem}
    A common variant of textbook RSA is the following: During key generation,
    the modulus $N$ is chosen as usual. We chose $e$ as $e:=3$ (instead of
    random). Then $d$ is chosen with $ed\equiv 1\pmod {\varphi(N)}$ (as usual).
    This is implemented by the Python functions \verb|rsa_keygen|,
    \verb|rsa_enc|, \verb|rsa_dec| below.
    
    We use this in a ``hybrid encryption'', which first picks an AES key $k$,
    encrypts it with RSA, and then encrypts the actual message with AES using
    the key $k$. (Functions \verb|hyb_enc|, \verb|hyb_dec|.)
    
    The  file  with the name {\tt 3rsa-aes-hybrid.py} on the lecture webpage
    contains these algorithms.
    
    Your task is to write an adversary that, given the public key \verb|pk|, and
    the hybrid encryption \verb|c| of some message \verb|m|, finds \verb|m|.
    That is, fill in the function body of the function \verb|adv| below so that
    the function \verb|test_adv| prints
    \verb|Success. The adversary broke the scheme|.
    
    Hint: Note that $k<\sqrt[3]N$ here. This makes it much simpler to find $k$
    given $k^3\pmod N$!
\end{problem}
\begin{solution}
    \textit{Implemented in code.}
    Part of RSA's security relies on the assumption that given well chosen
    values, there is no efficient algorithm for inverse modular exponentiation
    modulo semiprimes. That is, for large numbers, there is no efficient way to
    recover $m$ from $c=m^e \bmod N$, given only $N, e, c$, and where the prime
    factors $p, q$ of $N$ are not known.

    In the provided scheme, $N$ is the product of two $1024$-bit
    primes\footnotemark, and so $N$ is a $2048$-bit semiprime. We also have that
    $k$ is a $256$-bit number (AES key), and $e=3$. As such, for the RSA
    encryption of the AES key, we have $c = k^3 \bmod N$.
    \footnotetext{The code implementation actually generates $1025$-bit primes,
    and so $N$ is a $2049$-bit semiprime.}

    The catch is, that the largest number that can be represented by a $256$-bit
    number is $2^{256} - 1$. Let $l = k + 1 = 2^{256}$, and so
    $l^3 = \left(2^{256}\right)^3 = 2^{256*3} = 2^{768}$. Clearly $l^3 > k^3$.
    Since $N$ is a $2048$-bit number, $N \ge 2^{2047} > 2^{768} > k^3$, and so
    $c = k^3\bmod N = k^3$. As such, we can simply recover $k = \sqrt[3]{c}$,
    since fairly efficient algorithms exist for recovering cube roots. Then, we
    simply decrypt the AES encrypted message using the recovered symmetric AES
    key.
\end{solution}

\begin{problem}
    Recall that ElGamal can be defined with respect to many different groups.
    Here we give an example of a group one should not use.
    
    Let $p>0$ be a large prime. Let $G:=\{0,\dots,p-1\}$. The group operation is
    defined as follows: For $a,b\in G$, let $a\cdot b:=(a+b\bmod p)$.\footnote{%
    This is notationally highly confusing, of course, because it looks like we
    claim that plus and times are the same thing. But keep in mind that we are
    defining a new operation on $G$ here, and it is just a notational convention
    that we write it like multiplication. In particular, do not confuse $a\cdot
    b$ (the group operation) with $a\cdot b\bmod p$ (actual multiplication
    modulo $p$). Often one would use the symbol $+$, but that would be confusing
    as well because in our definitions of ElGamal we used multiplicative
    notation. If you wish, you can introduce a different symbol for the group
    operation, say $\circ$, and then the definition becomes $a\circ b:=(a+b\bmod
    p)$. You are free to do it either way in your solution, but make sure that
    you do not mix up the different meanings of $a\cdot b$!}
    Recall that $a^i$ for $i\in\mathbb{N}$ is defined as
    $a^i:=a\cdot a\cdot a\cdot a\cdot a\cdot \dots\cdot a$ ($i$-times).
    \begin{enumerate}
    \item What is $a^i$ written in terms of $+$? (I.e., when unfolding the
    definition of the group operation, and using modular arithmetic.) This
    should be quite a simple operation!
     
    \item\label{dlog} Show that there is an efficient algorithm for solving the
    discrete logarithm problem in $G$. That is, given $a\in G$ and $b:=a^i\in G$
    (but not given $i$), the algorithm should compute $i$.
    
    Note: You can use, without proof, the fact that there is an efficient
    algorithm (Extended Euclidean Algorithm, EEA) that, given $p$ and
    $x\in\{1,\dots,p-1\}$ computes $y$ with $xy\equiv 1\mod p$. (This is
    multiplication modulo $p$, not the group operation. The fact that inverting
    works for all $x$ uses that $p$ is prime.)
    
    \item\label{breakelgamal} Show that the ElGamal encryption scheme built on
    $G$ is not secure. More specifically, given the public key and an encryption
    of a message, find that message. 
    \end{enumerate} 
\end{problem}
\begin{solution}
    \begin{enumerate}
        \item For addition in modular arithmetic, we have the property
        $$(a + b) \bmod c = [(a \bmod c) + (b\bmod c)] \bmod c.$$

        Therefore we have $a\cdot b = (a + b) \bmod p = [(a\bmod p) +
        (b\bmod p)] \bmod p$. If we choose $b = a$, then $a\cdot a = a^2 = (a +
        a)\bmod p$. It follows that
        \begin{align*}
            a^3 = a^2 \cdot a &= [((a + a)\bmod p) + a] \bmod p\\
            &= [(((a + a)\bmod p)\bmod p) + (a \bmod p)]\bmod p\\
            &= [((a + a)\bmod p) + (a \bmod p)]\bmod p\\
            &= [(a + a) + a]\bmod p = (a + a + a) \bmod p.
        \end{align*}

        More generally, for $i \neq 0$, we have
        \begin{equation}\label{eq:powdef}
            a^i := \left(\sum_{k = 1}^{i} a\right)\bmod p
            = ia \bmod p,
        \end{equation}
        where ``$ia\bmod p$'' denotes multiplication of modulo $p$ of $i$ and
        $a$.
        \item From $b := a^i$ and definition \eqref{eq:powdef} we have that $ia
        \equiv b \pmod{p}$, with respect to multiplication modulo $p$.

        If we are able to compute $a^{-1}$ such that $aa^{-1}\equiv 1\pmod{p}$,
        we have $ba^{-1}\equiv iaa^{-1} \equiv i \pmod{p}$ and we can recover
        $i$. Using the EEA, we can indeed efficiently calculate $a^{-1}$ because
        we are given $a, b, p$. There is hence an efficient algorithm for
        solving the (additive) discrete logarithm problem in $G$, since the EEA
        is efficient, and modular multiplication is efficient.\qed

        \item Let us remind ourselves of the generic group based ElGamal
        cryptosystem. In a group $G$ of order $p$ with $p$ prime, pick a
        generator $g$ of $G$. Then, for a secret value $x\in G$, compute the
        public value $h = g^x$. The values $p, G, g, h$ are public.
        
        To encrypt a message $m\in G$, pick uniformly at random $y\ursample G$,
        and form the tuple $(c_1,c_2) := (g^y, mh^y)$. In the given group, we 
        have $(g^y, mh^y) := (yg \bmod p, (m(yh \bmod p))\bmod p)$.
        
        We show that in the given group $G$ we can recover a message without
        knowing the secret $x$, though the same attack could be used to recover
        the secret $x$ and enable recovery of all messages.

        We showed in (2.) that there is an efficient algorithm for solving the
        DLP in $G$. Since $g$ is public\footnotemark, we can efficiently find
        $g^{-1}\in G$ s.t. $gg^{-1}\equiv 1\pmod{p}$ and then $c_1g^{-1} \equiv
        ygg^{-1} \equiv y \pmod{p}$.
        
        We can then compute using EEA the modular inverse $(yh)^{-1}$ of $yh$
        (the multiplication modulo $p$ of $y$ and $h$). As a result, we have
        $c_2(yh)^{-1} \equiv m(yh)(yh)^{-1} \equiv m \pmod{p}$ and we have
        recovered the message without the secret key.\qed

        Let us illustrate this with an example in a small group with $p = 11$,
        and $g=2$. We pick $x = 7$ and so
        $h = g^x = 2^7 := (2 + 2 + 2 + 2 + 2 + 2+ 2) \bmod 11 =3$. We then pick
        the message $m = 5$ and the ephemeral randomness $y = 9$ and encrypt the
        message. The resulting ciphertext is $(c_1, c_2) := (yg \bmod p,
        (m(yh \bmod p))\bmod p) = (18\bmod 11, (5 (27 \bmod 11)) \bmod 11) =
        (7, 25 \bmod 11) = (7, 3)$.

        We have $g^{-1}\equiv 2^{-1} \equiv -5 \equiv 6 \pmod{11}$, and so
        $y\equiv c_1g^{-1}\equiv 42 \equiv 9 \pmod{11}$ and we correctly
        recovered $y = 9$. We can then calculate $yh \equiv 27\equiv 5\pmod{11}$
        and so $(yh)^{-1} \equiv -2 \equiv 9 \pmod{11}$.

        Finally $m\equiv c_2(yh)^{-1} \equiv 27 \equiv 5\pmod{11}$ and we
        correctly recovered the message $m = 5$.

        \footnotetext{The public key is the tuple $(G, g, h)$.}
    \end{enumerate}
\end{solution}

\begin{problem}
    We note that in RSA, we use the values  $d,e$ and $N$ for encrypting and
    decrypting but not the factors of $N$ --- thus, in essence, we could  devise
    a scheme where some central authority picks the values $p$ and $q$, computes
    $N=pq$ and $\varphi(N)=(p-1)(q-1)$, and uses  those to randomly sample
    $e_1,e_2,\dots, e_k$ and obtain pairs of keys $(d_1,e_1), (d_2,e_k),\dots,
    (d_k,e_k)$. Then the authority could distribute the $d_1$ to party $P_1$,
    $d_2$ to $P_2$ etc and publish the $e_i$ as the public keys. Then each party
    $P_i$ could use $d_i$ as their secret key. In addition, because everyone is
    operating on the same algebraic structure --- integers modulo $N$ --- there
    would be gains in efficiency.
    
    This scenario has been described as a Python program below that has been
    also uploaded as  \texttt{shared-n-rsa.py} to the course webpage. Your task
    is to write an adversary  \texttt{adv1} in that program, that, given two
    pairs of keys $(d_1,e_1)$ and $(d_2,e_2)$ as well as the public parameters,
    is able to factor $N$ with a reasonable probability. This describes the
    scenario where two parties collaborate and are able to obtain the secret
    keys of other parties.
    
    Your adversary does not have to succeed every time but should at least be
    able to occasionally succeed and not take too much time. You can pick how
    many tests you want to run, if you obtain at least one success with the
    total running time of a few minutes, this will be considered a valid
    solution. Use the function \texttt{testSharedNRSA} for testing, however, do
    use $50$ for the value of \texttt{bits} in
    \texttt{testSharedNRSA(bits,times,adv)}.
    
    Note: It is in fact possible to factor $N$ knowing only one pair
    $(d_1,e_1)$, however, this is more complicated.
\end{problem}
\begin{solution}
    
\end{solution}
\end{document}
